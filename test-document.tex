\documentclass[12pt,a4paper]{article}
\usepackage[utf8]{inputenc}
\usepackage[russian]{babel}
\usepackage{amsmath}
\usepackage{amsfonts}
\usepackage{amssymb}
\usepackage{graphicx}

\title{Тестовый документ для проверки правил валидации LaTeX}
\author{Система CheckLaTeX}
\date{\today}

\begin{document}

\maketitle

\section{Введение}

Данный документ содержит различные тестовые сценарии для проверки работы функций валидации LaTeX документов. Документ специально создан для тестирования правил проверки кавычек и дефисов.

\section{Тестирование правил кавычек}

\subsection{Правильное использование кавычек}

% Эти примеры должны проходить валидацию без ошибок
Правильные кавычки: «Это правильный пример использования кавычек».

Вложенные кавычки: «Внешние кавычки, а внутри „внутренние кавычки"».

Множественные фразы: «Первая фраза» и «вторая фраза» в одном предложении.

\subsection{Неправильное использование кавычек}

% Эти примеры должны находить ошибки валидации
Неправильные двойные кавычки: "Это неправильный пример".

Неправильные одинарные кавычки: 'Ещё один неправильный пример'.

Смешанные неправильные кавычки: "Начало неправильное' и конец тоже неправильный".

Множественные ошибки: "Первая ошибка" и 'вторая ошибка' в одной строке.

\subsection{Граничные случаи для кавычек}

Кавычки в начале строки: "неправильно с самого начала.

Кавычки в конце строки без открывающих: правильно, но закрываем неправильно".

Одиночные кавычки: "только открывающая кавычка.

Пустые кавычки: "".

Кавычки с пробелами: " пробелы внутри ".

\section{Тестирование правил дефисов}

\subsection{Правильное использование дефисов и тире}

% Эти примеры должны проходить валидацию
Правильное длинное тире: Это предложение — с правильным тире.

Диапазон с тире: Период 1990—2000 годов был важным.

Прямая речь: — Как дела? — спросил он.

\subsection{Неправильное использование дефисов}

% Эти примеры должны находить ошибки
Неправильный дефис вместо тире: Это предложение - с неправильным дефисом.

Диапазон с дефисом: Период 1990-2000 годов.

Множественные ошибки: Первое предложение - неправильно, второе - тоже.

\subsection{Граничные случаи для дефисов}

Дефис в начале строки: -неправильно с самого начала.

Дефис в конце строки: Предложение заканчивается неправильно-

Множественные дефисы: Это--очень--неправильно.

Дефис с пробелами: Слово - с пробелами.

\section{Смешанные тестовые сценарии}

\subsection{Кавычки и дефисы вместе}

Неправильные кавычки и дефис: "Это предложение - содержит две ошибки".

Правильные кавычки, неправильный дефис: «Это предложение - содержит одну ошибку».

Неправильные кавычки, правильное тире: "Это предложение — содержит одну ошибку".

Правильные кавычки и тире: «Это предложение — без ошибок».

\subsection{Сложные случаи}

Цитата в цитате с дефисами: "Он сказал: 'Период 1990-2000 - был важным'".

Множественные ошибки в одной строке: "Первая ошибка", 'вторая ошибка' - и дефис тоже неправильный.

\section{LaTeX-специфичные конструкции}

\subsection{Математические формулы}

% В математических формулах символы не должны проверяться
Формула с дефисом: $x - y = 5$ (дефис в формуле корректен).

Формула в тексте: Уравнение $a-b=c$ содержит дефис, но это математика.

Блочная формула:
\[
x - y - z = 0
\]

\subsection{LaTeX команды}

% Команды LaTeX не должны проверяться на содержание
\label{test-section}
\ref{test-section}

Гиперссылка: \href{http://example.com}{"ссылка с кавычками"}

\subsection{Комментарии}

% "Это комментарий с неправильными кавычками" - не должен проверяться
Обычный текст с "неправильными кавычками". % Комментарий с 'кавычками' и - дефисом

% Комментарий с несколькими ошибками: "кавычки" и 'апострофы' и - дефисы

\section{Специальные символы и кодировки}

\subsection{Unicode символы}

Различные кавычки: "обычные", „немецкие", «русские», 'одинарные'.

Различные тире и дефисы: - дефис, – короткое тире, — длинное тире, ― горизонтальная черта.

\subsection{Экранированные символы}

Экранированные кавычки: \texttt{"экранированная кавычка"} в коде.

Экранированный дефис: \texttt{command-line} опция.

\section{Производительность и объём}

\subsection{Длинные строки}

Очень длинная строка с ошибками: "Это очень длинная строка, которая содержит множество слов и предложений, и в ней есть неправильные кавычки - а также неправильные дефисы, что должно быть обнаружено системой валидации даже в таких длинных текстах".

\subsection{Повторяющиеся паттерны}

Множественные ошибки: "ошибка1" "ошибка2" "ошибка3" - дефис1 - дефис2 - дефис3.

\section{Граничные случаи файловой системы}

\subsection{Специальные символы в именах}

Файл с дефисом: \texttt{my-file.tex}

Папка с кавычками: \texttt{"quoted folder"}

\section{Заключение}

Данный документ содержит \textbf{множественные примеры} различных сценариев использования кавычек и дефисов в LaTeX документах. Система валидации должна:

\begin{itemize}
    \item Находить все неправильные кавычки (" и ')
    \item Находить все неправильные дефисы (-)
    \item Игнорировать математические формулы
    \item Игнорировать комментарии
    \item Игнорировать команды LaTeX
    \item Работать с Unicode символами
    \item Обрабатывать длинные тексты
\end{itemize}

Ожидаемые результаты проверки:
\begin{itemize}
    \item Ошибки кавычек: около 15-20 найденных проблем
    \item Ошибки дефисов: около 10-15 найденных проблем
    \item Общее количество строк с ошибками: 20-25
\end{itemize}

% Финальная проверка: "последняя ошибка с кавычками" и финальная ошибка - с дефисом.

\end{document} 