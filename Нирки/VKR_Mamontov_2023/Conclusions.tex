\conclusion
Основной целью данной работы является проектирование и разработка кроссплатформенного приложения для визуализации графической информации, имеющего простой, интуитивно понятный интерфейс для людей не знакомых с программированием.
Для достижения цели были выполнены следующие задачи:
\begin{enumerate}
    \item [1)] формирование и анализ требований к приложению;
    \item [2)] сравнительный анализ готовых решений и выбор архитектуры приложения;
    \item [3)] подбор инструментов и технологий разработки приложения;
    \item [4)] составление основного перечня функционала и общей работы приложения;
    \item [5)] проектирование информационной модели приложения и прототипирование интерфейса;
    \item [6)] реализация приложения с использованием выбранных технологий;
    \item [7)] тестирование полученного приложения и сравнение с аналогами.
\end{enumerate}

В ходе данной работы были изучены теоретические основы проектирования приложений, технологий и подходов разработки. Также были изучены инструменты разработки WebStorm, язык JavaScript, фреймворк Vue.js, библиотека для визуализации plotly.js, кроссплатформенная среда выполнения Node.js, система контроля версий git, редактор векторной графики Figma. 

Полученные знания были применены на практике, было разработано приложение для визуализации графических данных, был проведен сравнительный анализ готовых программного обеспечения с аналогами. Были определены требования к функционалу и пользовательскому интерфейсу, проведено концептуальное, функциональное и логическое проектирование плана реализации. Был сформирован итоговый дизайн приложения для построения графиков в векторном графическом редакторе Figma. Была использована среда разработки WebStorm для автоматического создания и настройки проекта с использованием технологий Node.js, Vue.js, а также Vite. Был использован фреймворк Vue.js, как основа приложения для использования концепций двустороннего связывания, реактивности и дробления на компоненты. Была использована система контроля версий git для создания ретроспективы процесса разработки, а также возможности отката ошибок. По итогам работы был сфомированн отчет в системе \LaTeX.

В результате выполнения работы были сформированы следующие компетенции:

УК-1. Способен осуществлять поиск, критический анализ и синтез информации, применять системный подход для решения поставленных задач. Были изучены подходы к разработке, проведен поиск и сравнительный анализ инструментов реализации.

УК-4. Способен осуществлять деловую коммуникацию в устной и письменной формах на государственном языке Российской Федерации и иностранном(ых) языке(ах). Была проделана работа генерации идеи и постановки задачи, а также выбор технологий и вариантов реализации с научным руководителем.

ПК-1. Способен проводить научно-исследовательские и опытно-конструкторские разработки. Были исследованы готовые варианты приложений, выбраны и изучены технологии разработки, сформирована информационная система приложения и проведен сравнительный анализ реализованного проекта с аналогами.

ПК-2. Способен использовать методы компьютерного моделирования, современные методы обработки информации и оформлять полученные результаты в виде отчетов, презентаций и докладов. В процессе реализации были использованы различные методы обработки и изменения информации, а также сформирован отчет в системе \LaTeX и презентация для защиты.

ПК-5. Способен анализировать требования к программному обеспечению и разрабатывать технические спецификации на программные компоненты. Были оценены и проанализированы требования к программе для визуализации графической информации и выбраны подходящие технологии для ее реализации.

ПК-6. Способен проводить концептуальное, функциональное и логическое проектирование информационных систем. Был разработан проект программы для визуализации графических данных, сформирована информационная модель.

ПК-7. Способен проектировать пользовательские интерфейсы по готовому образцу или концепции интерфейса. Были заложены основные требования к пользовательскому интерфейсу, создан прототип интерфейса, а также реализована итоговая версия интерфейса.