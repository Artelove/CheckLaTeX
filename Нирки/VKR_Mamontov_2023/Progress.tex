\chapter*{Ход выполнения практики}
\addcontentsline{toc}{chapter}{Ход выполнения практики}

\section*{Задания и отметки о выполнении}

\task{Написать обзор по предметной области исследования,
основываясь на научной, учебной и учебно-методической литературе, как на русском, так
и на английском языках.}

\task{Изучить теоретические материалы связанные с визуализацией данных, подобрать инструменты разработки и освоить их документацию.}

\task{Провести анализ требований к функционалу программы для построения графиков.}

\task{Спроектировать программу для визуализации числовых данных, создать информационные модели, выбрать набор технологий для реализации и прототипировать пользовательский интерфейс.}

\task{Реализовать программу для визуализации числовых данных, используя выбранный набор технологий, провести тестирование и сравнительный анализ с другими приложениями сходного функционала.}

\task{Подготовить отчет по практике в соответствии с требованиями нормоконтроля.}

\task{Создать презентацию в соответствии с данным отчетом и устный доклад для публичной защиты.}

\clearpage
\section*{Отзыв о работе обучающегося}

В ходе выполнения производственной практики, преддипломной практики Мамонтов С.В. изучил теоретические основы проектирования программного обеспечения, визуализации данных, методы построения графиков, виды интерполяции данных, архитектуры и информационные модели приложений. Сергей Викторович спроектировал и реализовал программу для визуализации числовых данных. Подготовил отчет в соответствии с требованиями нормоконтроля в издательской системе \LaTeX. Успешно освоил компетенции, предусмотренные учебным планом.

Мамонтов Сергей Викторович полностью справился с поставленными перед ним задачами. Замечания к работе студента отсутствуют. Считаю, что его работа заслуживает оценки <<отлично>>.

\progressapprov
