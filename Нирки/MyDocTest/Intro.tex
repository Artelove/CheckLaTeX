\intro
Производственная практика, научно-исследовательская работа посвящена процессу разработки программного обеспечения для статического анализа кода системы компьютерной верстки \TeX. Целью работы является ознакомится с имеющейся информацией о целях и задачах статического анализа кода, выявить его преимущества и недостатки, проанализировать методы используемые для анализа, и, используя данную теоретическую базу подготовить и описать проект разработки ПО.

 В данной работе описана целесообразность использования статического анализа кода, а также затронуто построение файлов \LaTeX и его основных функций. Данные знания легли в основу практической работы, где был проработан файл конфигурации статического анализа и функции проверки анализируемых характеристик.

При создании любой системы раннее обнаружение и устранение ошибок значительно облегчает дальнейшую работу. В разработке ПО кроме модульного и функционального тестирования для повышения качества продукта могут применяться практики статического анализа кода, которые являются самым простым и эффективным способом предотвращения дефектов и выявления несоответствий исходного кода принятому стандарту оформления. Статический анализ можно рассматривать как автоматизированный процесс обзора кода. Инструменты статического анализа непрерывно обрабатывают исходные тексты программ и выдают программисту рекомендации обратить повышенное внимание на определенные участки кода.
