\chapter*{Ход выполнения практики}
\addcontentsline{toc}{chapter}{Ход выполнения практики}

\section*{Задания и отметки о выполнении}

\task{Написать обзор по предметной области исследования,
основываясь на научной, учебной и учебно-методической литературе, как на русском, так
и на английском языках. Необходимо и спользовать современную литературу на английском
и русском языках по тематике разработка программного обеспечения для статического анализа кода системы компьютерной верстки TeX, поиск которой можно осуществлять по библиографическим базам
Scopus, WoS, elibrary, ResearchGate, ADS, ЭБС Лань и др.}

\task{Создание и описание модели конфигурационного файла испольуемого для статического анализа документа.}

\task{Описать алгоритмы выполнения функций проверки характеристик конфигурацонного файла.}

\task{Составить и описать блок-схему работы спроектированной программы.}

\task{Подготовить отчет по практике в соответствии с требованиям к выполнению учебой практики, проектно-технологической практики.}

\task{Создать презентацию в соответствии с данным отчетом и устный доклад для публичной защиты.}

\clearpage
\section*{Отзыв о работе обучающегося}

В ходе выполнения учебой практики, проектно-технологической практике Карагичев А.В. изучил цели из задачи статического анализа кода, проанализировал методы используемые для анализа, и реализовал описание проекта разработки программного обеспечения. 
Подготови очет в соответствии с требованиями номкотроля в издательстве системе \LaTeX. Успешно освоил компетенции, предусмотренные учебным планом.
Карагичев Александр Владимирович полностью справился с поставленными перд ним задачами. Замечания к работе студента осутствуют. Считаю, что его работа заслуживает оценки \guillemotleftотлично\guillemotright.

\progressapprov
