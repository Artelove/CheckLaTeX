\conclusion
Основной целью данной работы являлось проектирование и описание функциональных возможностей программного обеспечения для статического анализа кода системы компьютерной верстки TeX, выдающей в качестве результата информационные сообщения, которые помогут исправить моменты несоответствия оформления работы относительно проверяемого конфигурационного файла. Для достижения цели были выполнены следующие задачи:
    \begin{enumerate}    
    \item выявление анализируемых характеристик;    
    \item создание и описание модели конфигурационного файла.
    \item описание алгоритма выполняемых функций проверки характеристик!   
    \item Cоставление хода работы программы;
    \end{enumerate}

В ходе данной работы были изучены материалы, связанные со статическим анализом, включая цели, задачи, преимущества и недостатки, была рассмотрена концепция написания документов с помощью системы верстки \TeX, было осуществлено ознакомление с основными командами \LaTeX, их аргументами и опциями. 

Изученная теория применена на практике, а именно – создан и описан конфигурационный файл являющийся шаблоном, по которому необходимо выполнять статический анализ, а так же описаны механизмы его проведения. А так же выполнено описание хода работы и алгоритмов функций програмного обеспечения.

По итогам работы был сфомированн отчет в системе \LaTeX.
В результате выполнения работы были сформированы следующие компетенции: 

При выполнении работы был проведен поиск нужной информации по теме «Разработка программного обеспечения для статического анализа кода системы компьютерной верстки TeX» и ее анализ,  что соответствует компетенции: 

УК-1. Способен осуществлять поиск, критический анализ и синтез информации, применять системный подход для решения поставленных задач.

Была проделана работа генерации идеи и постановки задачи, а также
выбор технологий с научным руководителем, что соответствует компетенции:

УК-4. Способен осуществлять деловую коммуникацию в устной и письменной формах на государственном языке Российской Федерации и иностранном(ых) языке(ах).

Были оценены и проанализированы требования к исходным данным и программе статичего анализа системы верстки \TeX\verb| |и выбраны подходящие технологии для ее реализации, что соответствует компетенции:

ПК-5. Способен анализировать требования к программному обеспечению и разрабатывать технические спецификации на программные компоненты.

Был разработан и описан проект программы для статического анализа системы верстки \TeX, что соответствует компетенции:

ПК-6. Способен проводить концептуальное, функциональное и логическое проектирование информационных систем.

Были заложены основны и требования к пользовательскому интерфейсу, а также подобраны технологии, для его реализации, что соответствует компетенции:

ПК-7. Способен проектировать пользовательские интерфейсы по готовому образцу или концепции интерфейса.