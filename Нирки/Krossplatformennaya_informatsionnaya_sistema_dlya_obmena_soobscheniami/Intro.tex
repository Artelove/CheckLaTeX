\intro

На текущий момент, с активным ростом скорости передачи информации внутри информационно-коммуникационных сетей, спрос на обмен информацией в различных сферах деятельности, таких как образование, бизнес, наука медицина, и т.д., растет. В связи с чем, важно иметь надежную и эффективную систему обмена сообщениями, обеспечивающую одинаковую функциональность на разных платформах.

Традиционно, все веб-сайты создавались на основе модели запрос/ответ, требуя определенных действий пользователя, для загрузки новой информации, с внедрением технологии AJAX в браузеры, появилась возможность выполнять обновление данных в реальном времени, с помощью цикличных запросов, что позволило создавать сложные веб-приложения с динамическим состоянием \cite{RealtimeWebAppPhp}. На данный момент, в браузерах появились технологии, позволяющие создавать полнодуплексные соединение между клиентом и сервером через одно TCP соединение, существенно упрощая обмен информацией между клиентом и сервером.

Использование веб-платформы для создания информационных систем обеспечивает ряд ключевых преимуществ. Благодаря повсеместному распространению интернета и открытой стандартизации веб-платформы, веб-приложения доступны с разных устройств и операционных систем, имеют модель безопасности, ориентированную на пользователя, при этом не существует единой компании, которая бы контролировала спецификацию и реализацию веб-платформы \cite{webofthings}.

Подход прогрессивных веб-приложений, активно набирающий популярность, с момента появления в 2015 году, позволяет интегрировать веб-приложение в операционную систему, тем самым совмещая открытость, безопасность и распространенность веб-платформы с функциональными возможностями платформо-зависимых приложений. Новые API-интерфейсы позволяют получать доступ к файловой системе, управлять мультимедиа, использовать push-уведомления, использовать аппаратные датчики, управлять сетью, выполнять инсталляцию веб-приложения в операционную систему.

В ходе данного научного исследования рассмотрена веб-платформа, как открытая, кроссплатформенная среда для создания информационных систем.  Исследован подход прогрессивных веб-приложений, в совокупности с новыми API-интерфейсами веб-платформы, позволяющий реализовать функционал платформо-зависимых приложений. Рассмотрены методы, технические средства и протоколы, позволяющее осуществлять обмен информацией в режиме реального времени, проанализирована их эволюция и перспективы дальнейшего развития. Были использованы современные DevOps-практики для улучшения опыта разработки и обеспечения бесперебойной работы информационной системы, за счет обеспечения непрерывной интеграции благодаря сервису Gitlab CI, использования контейнеризации, реализованной посредством инструментария Docker при развертывании системы. Разработан безопасный протокол обмена информацией между клиентом и сервером поверх протокола WebSocket. Спроектирована и реализована серверная часть информационной системы с использованием фреймворка Nest.js. Спроектирована и реализована клиентская часть информационной системы с использованием фреймворка Vue.js.

Объектом данного исследования является кроссплатформенная информационная система для обмена сообщениями, разработанная на базе веб-технологий, обеспечивающая коммуникацию между пользователями.

Предметом исследования являются особенности, методы и средства создания кроссплатформенной информационной системы для обмена сообщениями. В частности, предметом исследования является исследование веб-платформы, подхода прогрессивных веб-приложений, современных API-интерфейсов веб-платформы, методов и протоколов для обмена информацией в режиме реального времени, DevOps-практик, контейнеризации и безопасности веб-приложений. Также предметом исследования является проектирование и разработка серверной и клиентской частей информационной системы с использованием фреймворков Nest.js и Vue.js соответственно.

Актуальность данного исследования заключается в том, что информационные системы для обмена сообщениями позволяют существенно улучшить эффективность процесса коммуникации в различных сферах деятельности. В связи с чем, безопасность, доступность и удобство использования таких систем становится критически важным для обеспечения работы в этих сферах. Изначальный фокус на веб-платформе, делает информационную систему независимой от конкретной аппаратной платформы или операционной системы, что повышает ее устойчивость к блокировкам или ограничениям от корпораций, которые владеют операционными системами и магазинами приложений \cite{webofthings2}. Разработанные в ходе исследования подходы, протоколы, программные компоненты и дизайн-система, также могут быть переиспользованы в других информационных системах, существенно ускорив процесс их разработки.

Общая цель работы — исследовать этапы, методы и средства создания кроссплатформенной информационной системы на базе веб-технологий, спроектировать и осуществить программную реализацию информационной системы, которая обеспечивает безопасный обмен сообщениями между пользователями в режиме реального времени, с помощью изученных в результате исследования методов и средств. В ходе достижения общей цели научно-исследовательской работы решаются следующие задачи:

\begin{enumerate} 
  \item Исследовать возможности веб-платформы и подхода прогрессивных веб-приложений для создания кроссплатформенной информационной системы.
  
  \item Проанализировать методы, технические средства и протоколы, позволяющие осуществлять обмен информацией в режиме реального времени, проанализировать их эволюцию и перспективы дальнейшего развития.
  
  \item Исследовать и применить методы и практики DevOps для улучшения опыта разработки и обеспечения бесперебойной работы информационной системы, в том числе использование сервиса Gitlab CI и контейнеризации с помощью Docker при развертывании системы.

  \item Разработать безопасный протокол обмена информацией между клиентом и сервером поверх протокола WebSocket.

  \item Спроектировать информационную систему «Corpic», обеспечивающую безопасный обмен сообщениями между пользователями в режиме реального времени, включая такие этапы, как определение основных функциональных требований, определение архитектуры информационной системы, включая определение компонентов системы, выбор технологий, инструментов и программных средств их реализации, проектирование пользовательского интерфейса информационной системы.

  \item Выполнить программную реализацию информационной системы «Corpic», в соответствии с разработанным проектом, включая создание серверной части информационной системы с помощью фреймворка Nest.js, создание клиентской части информационной системы с помощью фреймворка Vue.js.

  \item Сделать выводы о достижении поставленной цели и оценить перспективы дальнейшего развития разработанной информационной системы.
\end{enumerate}