\conclusion
В ходе данной производственной практике, преддипломнай практике была исследована веб-платформа как открытая, кроссплатформенная среда для создания информационных систем, был рассмотрен подход прогрессивных веб-приложений (PWA) и новые API-интерфейсы веб-платформы для реализации функционала платформо-зависимых приложений, были исследованы современные методы и подходы для оптимизации и ускорения веб-приложений.

Был рассмотрено что такое коммуникация в режиме реального времени и приложения, работающие в режиме реального времени. Были рассмотрены основные категории систем реального времени в виде мягких и жестких систем. Были рассмотрены методы, технические средства и протоколы, позволяющее осуществлять коммуникацию в режиме реального времени, в частности, использование цикличных HTTP запросов, WebSocket и WebRTC, проанализирована их эволюция и перспективы дальнейшего развития.

Была рассмотрена организация процесса разработки информационной системы с использованием методологии DevOps, для улучшения опыта разработки и обеспечения бесперебойной работы информационной системы, в том числе использование сервиса Gitlab CI и контейнеризации с помощью Docker при развертывании системы.

Была спроектирована информационная система «Corpic», обеспечивающая безопасный обмен сообщениями между пользователями, в которой для обмена данными между клиентом и сервером используются рассмотренные технологии коммуникации в режиме реального времени. В том числе, был спроектирован пользовательский интерфейс информационной системы.

Была выполнена программная реализация информационной системы «Corpic», в соотвествии с разработанным проектом, включая создания серверной части информационной системы с помощью фреймворка Nest.js, создание клиентской части информационной системы с помощью фреймворка Vue.js. В ходе реализации информационной системы, были применены результаты исследования веб-платформы, в том числе подход прогрессивных веб-приложений, методы и подходы для оптимизации и ускорения веб-приложений.

Перечень компетенций для направления 09.03.02 Информационные системы
и технологии по производственной практике, преддипломнай практике:

УК-1. Способен осуществлять поиск, критический анализ и синтез информации, применять системный подход для решения поставленных задач. Компетенция была сформирована в процессе анализа и поиска современной литературы на английском и русском языках по тематике исследования, с помощью библиографических баз Scopus, WoS, elibrary, ResearchGate, ADS, ЭБС Лань, и др.

УК-4. Способен осуществлять деловую коммуникацию в устной и письменной формах на государственном языке Российской Федерации и иностранном(ых) языке(ах). Компетенция была сформирована в процессе деловой коммуникации с научным консультантом в письменной и устной формах по вопросам, касающихся учебной практики.

ПК-1. Способен проводить научно-исследовательские и опытно-конструкторские разработки. Компетенция была сформирована в процессе исследования веб-платформы и подхода прогрессивных веб-приложений, анализа протоколов обмена информацией в веб-платформе, исследования методов и практик DevOps для обеспечения бесперебойной работы информационной системы, программной реализации информационной системы «Corpic», в соответствии с разработанным проектом. 

ПК-2. Способен использовать методы компьютерного моделирования, современные методы обработки информации и оформлять полученные результаты в виде отчетов, презентаций и докладов. Компетенция была сформирована в процессе документирования протокола взаимодействия между сервером и клиентом, документирования API веб-сервера, оформления отчета по производственной практике, преддипломной практике с использованием набора макрорасширений LATEX, подготовки мультимедийной презентации и доклада по результатам производственной практике, преддипломной практике.

ПК-5. Способен анализировать требования к программному обеспечению и разрабатывать технические спецификации на программные компоненты. Компетенция была сформирована в процессе исследования методов, технических средств и протоколов, позволяющее осуществлять коммуникацию в режиме реального времени. Разработке архитектуры сетевого взаимодействия между клиентом и сервером на основе протокола WebSocket.

ПК-6. Способен проводить концептуальное, функциональное и логическое проектирование информационных систем. Компетенция была сформирована в процессе проектирования архитектуры серверной и клиентской части информационной системы «Corpic».

ПК-7. Способен проектировать пользовательские интерфейсы по готовому образцу или концепции интерфейса. Компетенция была сформирована в процессе проектирования пользовательского интерфейса веб-приложения. Разработки дизайн системы, включающей в себя цветовые схемы и компоненты с различными состояниями.


