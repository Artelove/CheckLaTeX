\chapter*{Ход выполнения практики}
\addcontentsline{toc}{chapter}{Ход выполнения практики}

\section*{Задания и отметки о выполнении}

\task{Осуществить поиск информации по предметной области исследования,
основываясь на научной, учебной и учебно-методической литературе, как на русском, так и на английском языках. Необходимо и спользовать современную литературу на английском и русском языках по тематике производственной практике, преддипломной практике, поиск которой можно осуществлять по библиографическим базам Scopus, WoS, elibrary, ResearchGate, ADS, ЭБС Лань и др.}

\task{Исследовать возможности веб-платформы и подхода прогрессивных веб-приложений для создания кроссплатформенной информационной системы.}

\task{Проанализировать методы, технические средства и протоколы, позволяющие осуществлять обмен информацией в режиме реального времени, проанализировать их эволюцию и перспективы дальнейшего развития.}

\task{Исследовать и применить методы и практики DevOps для улучшения опыта разработки и обеспечения бесперебойной работы информационной системы, в том числе использование сервиса Gitlab CI и контейнеризации с помощью Docker при развертывании системы.}

\task{Разработать безопасный протокол обмена информацией между клиентом и сервером поверх протокола WebSocket.}

\task{Спроектировать информационную систему «Corpic», обеспечивающую безопасный обмен сообщениями между пользователями.}

\task{Выполнить программную реализацию информационной системы «Corpic», в соответствии с разработанным проектом.}

\task{Сделать выводы о достижении поставленной цели и оценить перспективы дальнейшего развития разработанной информационной системы}

\task{Подготовить отчет по практике в соответствии с требованиями нормоконтроля кафедры ИСКМ. Оформить отчет по производственной практике, преддипломной практике с использованием набора макрорасширений LATEX.}

\task{Подготовить мультимедийную презентацию и доклад по результатам производственной практике, преддипломной практике.}

\clearpage
\section*{Отзыв о работе обучающегося}

В ходе выполнения производственной практике, преддипломной практике, Мостовой М. С. выполнил исследование возможностей веб-платформы и подхода прогрессивных веб-приложений для создания кроссплатформенной информационной системы. Студент рассмотрел различные методы и технологии, позволяющие осуществлять обмен информацией в режиме реального времени, выделил их достоинства и недостатки, проанализировал их эволюцию и перспективы развития. Изучил методы и практики DevOps для улучшения опыта разработки и обеспечения бесперебойной работы информационной системы. Разработал безопасный протокол обмена информацией между клиентом и сервером поверх протокола WebSocket. Студент выполнил проектирование информационной системы «Corpic», обеспечивающей безопасный обмен сообщениями между пользователями, включая проектирование пользовательского интерфейса. Выполнил программную реализацию информационной системы. Подготовил отчет в соответствии с требованиями нормоконтроля в издательской системе LATEX. Успешно освоил компетенции, предусмотренные учебным планом. Все задачи, поставленные перед студентом, были выполнены в срок. Считаю, что работа Мостового М. С. заслуживает оценки «отлично».

\progressapprov
