\conclusion
Заключение должно содержать перечисление результатов, полученных при
выполнении работы, а также те выводы, которые вы сделали при ее
выполнении. Также заключение может содержать предложения,
рекомендации и перспективы дальнейшего развития темы. 

Далее в заключении приводится перечень компетенций, освоенных вами
за время выполнения практики. Для каждой компетенции приводится её
формулировка и описание того, как именно вы ее освоили при
выполнении своей работы.

Перечень компетенций для направления 09.03.02 Информационные системы
и технологии по производственной практике, преддипломнай практике:

УК-1. Способен осуществлять поиск, критический анализ и синтез информации, применять системный подход для решения поставленных задач.

УК-4. Способен осуществлять деловую коммуникацию в устной и письменной формах на государственном языке Российской Федерации и иностранном(ых) языке(ах).

ПК-1. Способен проводить научно-исследовательские и опытно-конструкторские разработки.

ПК-2. Способен использовать методы компьютерного моделирования, современные методы обработки информации и оформлять полученные результаты в виде отчетов, презентаций и докладов.

ПК-5. Способен анализировать требования к программному обеспечению и разрабатывать технические спецификации на программные компоненты.

ПК-6. Способен проводить концептуальное, функциональное и логическое проектирование информационных систем.

ПК-7. Способен проектировать пользовательские интерфейсы по готовому образцу или концепции интерфейса.

