\intro
Научно-исследователская работа посвящена методам и средствам распознавания графических изображений.

Целью данной работы является проектирование, реализация и обучение сверточной нейронной сети проводящей дихотомию фотографий различных людей на 2 класса: фотографии с мужчинами и фотографии с женщинами. Для достижения этой цели необходимо выполнить следующие задачи:

\begin{enumerate}
    \item [1)] изучение теоретических материалов, посвященных разработке и обучении сверточных нейронных сетей;
    \item [2)] поиск данных для обучения;
    \item [3)] проектирование архитектуры нейронной сети;
    \item [4)] реализация на языке Python и обучение;
    \item [5)] проведение тестов и состовление отчета.
\end{enumerate}



В современных реалиах машинное обучение позволяет решать обширный
класс задач, которые практически невозможно решить стандартными методами
программирования. Эти задачи требуют сложных правил. которые трудно составить вручную, поэтому процесс составления правил перекладывается на работу
машины. Распознавание графических изображений одна из передовых технологий машинного обучения. В частности компьютерное зрение используется в создании систем управления процессамм для промышленных роботов или автономных транспортных средств, систем видеонаблюдения, взаимодействия и т.д.

