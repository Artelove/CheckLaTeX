\conclusion
В ходе данной работы были изучены теоретические основы, типы и задачи машинного обучения, было осуществленно ознакомление с языком программирования Python, библиотеками Keras, TensorFlow, Tkinter и Matplotlib. Был осуществлен поиск наборов для обучения в системе организации конкурсов по исследованию данных. Изученная теория была применена на практике, а именно была спроектрирована архитектура модели сверточной нейронной сети. Была написана реализция этой сети на языке Python и впоследствии обучена на наборе данных, содержащем 200000 фотографий различных мужчин и женщин. Была написана программа для разделениия картинок на 2 класса: фотографий на которых изображены мужчины и фотографии, на которых изображены женщины. Также во время разработки, отладки модели сети и написания интерфейса программы была изучена среда разработки PyCharm. Были произведены исследование и тестирование полученной программы, как на заранее подготовленном наборе тестовых данных, так и на произвольных данных из сети интернет. 

В результате выполнения работы сформированы следующие компетенции:

Был проведен поиск нужной информации по теме «Методы и средства распознования графических изображений» и ее анализ, что соотвествует компетенции:

УК--1 Способен осуществлять поиск, критический анализ и синтез информации, применять системный подход для решения поставленных задач.

Происходило взаимодействие с научным руководителем в устной и письменной формах по вопросам, которые касались реализации сверточной нейронной сети для бинарной классификации изображений, что соответсвует компетенции:

УК--4 Способен осуществлять деловую коммуникацию в устной и письменной формах на государственном языке Российской Федерации и иностранном(ых) языке(ах).

В ходе данной работы были изучены теоретические материалы и спроектирована, реализована и обучена сверточная нейронная сеть проводящая дихотомию фотографий различных людей на 2 класса, что соответствует компетенции:

ПК--1. Способен проводить научно-исследовательские и опытно-конструкторские разработки.

Проектирование и реализация нейронной сети были осуществлены на языке программирования Python в среде PyCharm, а также был составлен отчет в системе \LaTeX и презентация по всей проделанной работе, что соответсвует компетенции: 

ПК--2. Способен использовать методы компьютерного моделирования, современные методы обработки информации и оформлять полученные результаты в виде отчетов, презентаций и докладов.