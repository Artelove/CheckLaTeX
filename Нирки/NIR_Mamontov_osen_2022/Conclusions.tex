\conclusion
Основной целью данной работы является проектирование кроссплатформенного приложения для визуализации графической информации, имеющего простой, интуитивно понятный интерфейс для людей не знакомых с программированием.
Для достижения цели были выполнены следующие задачи:
\begin{enumerate}
    \item [1)] формирование и анализ требований к приложению;
    \item [2)] сравнительный анализ готовых решений и выбор архитектуры приложения;
    \item [3)] подбор инструментов и технологий разработки приложения;
    \item [4)] составление основного перечня функционала и общей работы приложения;
    \item [5)] проектирование информационной модели приложения и прототипирование интерфейса.
\end{enumerate}

В ходе данной работы были изучены теоретические основы проектирования приложений, технологий и подходов разработки. Также были изучены инструменты разработки WebStorm, язык JavaScript, фреймворк Vue.js. 

Полученные знания были применены на практике, был разработан проект приложения для визуализации графических данных, был проведен сравнительный анализ технологий и инструментов разработки. Были определены требования к функционалу и пользовательскому интерфейсу, проведено концептуальное, функциональное и логическое проектирование плана реализации. По итогам работы был сфомированн отчет в системе \LaTeX.

В результате выполнения работы были сформированы следующие компетенции:

Были изучены подходы к разработке, проведен поиск и сравнительный анализ инструментов реализации что соответствует компетенции 
УК-1. Способен осуществлять поиск, критический анализ и синтез
информации, применять системный подход для решения поставленных
задач.

Была проделана работа генерации идеи и постановки задачи, а также выбор технологий с научным руководителем, что соответствует компетенции 
УК-4. Способен осуществлять деловую коммуникацию в устной и
письменной формах на государственном языке Российской
Федерации и иностранном(ых) языке(ах).

Были оценены и проанализированы требования к программе для визуализации графической информации и выбраны подходящие технологии для ее реализации, что соответствует компетенции
ПК-5. Способен анализировать требования к программному обеспечению
и разрабатывать технические спецификации на программные компоненты.

Был разработан проект программы для визуализации графических данных, сформирована информационная модель, что соответствует компетенции
ПК-6. Способен проводить концептуальное, функциональное и логическое
проектирование информационных систем.

Были заложены основные требования к пользовательскому интерфейсу, создан прототип интерфейса, а также подобраны технологии, для его реализации, что соответствует компетенции
ПК-7. Способен проектировать пользовательские интерфейсы по готовому
образцу или концепции интерфейса.
