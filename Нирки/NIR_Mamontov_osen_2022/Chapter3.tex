\chapter{Оформление библиографии}\label{Bib_chapter}

Для оформления списка литературы в шабоне используется система Bib\TeX.  Программное обеспечение для создания форматированных списков библиографии Bib\TeX, позволяет упростить и (полу)автоматизировать процесс оформления списка литературы. 

Для использования Bib\TeX{} необходимо записать список источников, на которые вы ссылаетесь в своем тексте, в специальный файл с расширением .bib (Bib.bib). Bib\TeX{} файл состоит из записей, которые начинаются с символа \verb|@| и указанием типа записи. Что бы ни было записано в вашем bib-файле, Bib\TeX{} по умолчанию включает в список литературы только те источники, на которые вы ссылаетесь с помощью команды \verb|\cite|. Использовать команду \verb|\nocite| для включения записей, на которые нет ссылок в тексте, запрещается. Пример библиографической записи \cite{Jackson2021TheOO} в bib-файле приведен в листинге~\ref{ls:02}.

\begin{lstlisting}[caption={Пример библиографической записи}, label={ls:02}, language=TeX]
@article{Jackson2021TheOO,
  title={The origin of low-surface-brightness galaxies in the 
  dwarf regime},
  Ryan A. Jackson and Garreth Martin and Sugata Kaviraj 
  and Marius Ramsoy and Julien Devriendt and Thomas M. Sedgwick 
  and C. Laigle and H. Choi and Ricarda S Beckmann and Marta 
  Volonteri and Yohan Dubois and Christophe Pichon 
  and  Sukyoung K. Yi and Adrianne D. Slyz and Katarina Kraljic 
  and Taysun Kimm and S{\'e}bastien Peirani and Ivan K Baldry},
  journal={Monthly Notices of the Royal Astronomical Society},
  year={2021},
  volume={502},
  pages={4262-4276}
}
\end{lstlisting}

В первой строке, сразу после \verb|@article{|, стоит уникальная библиографическая метка \verb|Jackson2021TheOO|; при ссылках в тексте на указанную работу необходимо написать  \verb|\cite{Jackson2021TheOO}|. Далее идут поля записи, также разделенные запятыми  Смысл большинства полей ясен из их названия. 

Для заполнения \verb|bib|-файла можно воспользоваться библиографическими базами данных, которые позволяют выгрузить готовую запись в \verb|bib|-файл. Пример такой базы и экспорта цитирований приведен на рисунке~\ref{fig:BibTex} 

\begin{figure}[h!]
\begin{center}
\includegraphics[width=0.8\hsize]{bib-1.jpg}\\[2mm]
\includegraphics[width=0.8\hsize]{bib-2.jpg}\\[2mm]
\caption{Пример экспорта Bib\TeX-файла на примере системы SemanticScholar}\label{fig:BibTex}
\end{center}
\end{figure}

\newpage
\textbf{ВНИМАНИЕ!}

К списку источников предьявляютя особые требования. Это касается как его содержания, так и количества источников, на которые вы ссылаетесь в своем отчете. В первой строке таблицы~\ref{tab:lit} указан номер курса, для которого в строке ниже приведено минимальное общее количество источников, на которые есть ссылки в работе. В последующих строках приведено минимальное количество источников разных типов. Под научно-периодической литературой подразумевается наличие ссылок на статьи в периодических тематических журналах.

\begin{table}[h]
\caption{Название таблицы. Таблицы следует размещать в основном тексте рядом с первым цитированием.}
\label{tab:lit}
\begin{center}
\begin{tabular}{|p{7.5 cm}|c|c|c|c|c|}
\hline
 Номер курса & 2 & 3 & 4 & 5 & 6\\
\hline
Общее количество используемой литературы из них: & > 15 & > 20 & > 25 & > 30 & > 30 \\
\hline
- на иностранных языках & $\ge$4 & $\ge$5 & $\ge$6 & $\ge$7 & $\ge$7 \\
\hline
- текущая научно-периодическая литература (после 2010 г.)& $\ge$3 & $\ge$5 & $\ge$7 & $\ge$7 & $\ge$7  \\
\hline
- литература 21 века & $\ge$10 & $\ge$15 & $\ge$21 & $\ge$21 & $\ge$21 \\
\hline
\end{tabular}
\end{center}
\end{table}

\chapter{Элементы, которые могут присутствовать в тексте}

\section{Формулы}

Для оформления формул и уравнений используются стандартные возможности \LaTeX.
Выключные формулы должна быть оформлены как окружение \verb|equation|. Пояснения символов и числовых коэффициентов, 
входящих в формулу, если они не пояснены ранее в тексте,
должны быть приведены непосредственно под формулой. Пояснение каждого
символа следует давать в той последовательности, в которой
символы приведены в формуле. Первая строка пояснения должна начинаться
со слова  \guillemotleftгде\guillemotright\ без двоеточия после него.
Формулы, следующие одна за другой и не разделенные текстом, разделяют запятой

Ниже приведен простейший пример
\begin{equation}\label{eq:H0}
H_0 = \frac{p^2}{2 m} + a_0\, x^4 - a_2\, x^2,
\end{equation}
где  $p$ -- импульс частицы, $m$ -- масса частицы, $x$ -- координата частицы, $a_0, a_2$ -- параметры системы.

Нумерация формул осуществляется автоматически в пределах главы, как это
продемонстрировано для формулы (\ref{eq:H0}).
Переносить формулы на следующую строку допускается только на знаках выполняемых операций,
причем знак в начале следующей строки повторяют. 

Ссылки в тексте на порядковые номера формул дают в скобках, напри-
мер, . . . в формуле (\ref{eq:H0}).

А это уже более сложный пример
\begin{align}
H^0_{m\, n} & = \delta_{m + 4 \; n} \;\frac{a_0 g^2}{4} \sqrt{(m + 1)(m + 2)(m +
3)(m +
4)} +\notag\\
 & + \delta_{m + 2 \; n} \;\frac{g}{2} a_0  (2 m + 3) \sqrt{(m
+ 1)(m  + 2)}.\label{eq:Hmn}
\end{align}
где $g = \frac{m}{a_0}$ -- числовой коэффициент, или так $\displaystyle{g = \frac{m}{a_0}}$, если поместить внутристрочную формулу в аргумент команды \verb|\displaystyle{}|.



\section{Рисунки}

Для вставки рисунков рекомендуется использовать пакеты graphics или graphicx и команду
\verb|\includegraphics|. Рисунок оформляется с использованием стандартного окружения \verb|figure|, которое обеспечивает его автоматическую нумерацию. Рисунок и подпись вставляются при помощи стандартных команд \verb|\includegraphics| и \verb|\caption{Подпись}| соответственно. Подрисуночная подпись должна пояснять что изображено на рисунке, также может содержать уточняющую информацию. 

Рисунки должны быть пронумерованы и иметь подпись, которая всегда должна располагаться 
под рисунками. Рисунки могут быть представлены в формате * .pdf, * .ps, * .eps, * .jpg, 
* .png или * .tif и должны иметь разрешение не менее 300 dpi. 

Убедитесь, что линии на рисунках не прерываются и имеют постоянную ширину. Сетки и детали
на рисунках должны быть четкими и не должны быть начертаны друг на друге. Аббревиатуры, 
используемые на рисунке, должны быть определены в тексте отчета, если они не являются общими сокращениями или уже были определены в тексте. Легенда должна пояснять все используемые символы и должна входить с состав рисунке, а не содержаться в словесных пояснениях в подписях (например, <<пунктирная линия>> или <<открытые зеленые кружки>>).

На все рисунки в тексте должна быть ссылка до появления самого рисунка в тексте отчета, например: рисунки \ref{fig:01}, \ref{fig:02}.

\begin{figure}[h!]
\begin{center}
\includegraphics[angle=270,width=10cm]{test}\\[2mm]
\caption{Подпись к рисунку. Дополнительная информация. Дополнительная инофрмация}\label{fig:01}
\end{center}
\end{figure}

\begin{figure}[h!]
\begin{center}
\includegraphics[width=0.5\hsize]{empty.jpg}\\[2mm]
\caption{Пример использования *.jpg}\label{fig:02}
\end{center}
\end{figure}

Если возникает необходимость привести в тексте рисунок, который не помещается на одной странице, например, это может быть большая блок-схема, то номер рисунка не должен изменяться, а в конце подписи к нему должно быть указано: Лист 1. На следующей странице должен быть приведен только номер рисунка, а подпись должна содержать только Лист 2.
Например: Рисунок 2.1 -- Лист 2.


\section{Таблицы}

Для создания таблиц используется окружение \verb|table|, которое обеспечивает 
нумерацию и создание заголовка. Для помещения заголовка над
таблицей в начале указанного окружения необходимо задать команду
\verb|\capition{Подпись к таблице}|. Сама таблица может оформляться с помощью стандартного окружения \verb|tabular|. Пример таблицы приведен ниже (таблица \ref{tab:1}). 

\begin{table}[h]
\caption{Название таблицы. Таблицы следует размещать в основном тексте рядом с первым цитированием.}
\label{tab:1}
\begin{center}
\begin{tabular}{|l|c|c|}
\hline
Заголовок 1 & Заголовок 2 & Заголовок 3 \\
\hline
Запись 1 &данные &данные\\
\hline
Запись 2 &данные &данные\\
\hline
Запись 3 &данные &данные\\
\hline
\end{tabular}
\end{center}
\end{table}

На все таблицы документа должны быть приведены ссылки в тексте
документа до появления самой таблицы. При ссылке следует писать слово \guillemotleftтаблица\guillemotright\ с указанием её номера.

В случае наличия в тексте длинных таблиц, которые не помещаются на одной странице, необходимо использовать окружение \verb|longtable|.

\section{Листинги программ}

Листинги программ могут быть расположены как в тексте отчета (возможно ближе к 
соответствующим частям текста, содержащим ссылку на листинг),  так и в конце его текста отчета (в приложениях). Листинг, за исключением листингов приложений, следует нумеровать арабскими цифрами с нумерацией в пределах главы (листинг \ref{ls:01}).
Листинг каждого приложения обозначают отдельной нумерацией арабскими цифрами с добавлением перед цифрой обозначения приложения. Например: Листинг \ref{ls:a:01}.
На все листинги должны быть ссылки в тексте отчета до появления самого листинга, примеры таких ссылок приведены выше.

\begin{lstlisting}[caption={Пример листинга в тексте}, label={ls:01}]
// a simple kernel that simply increments each array element by b
__global__ void kernelAddConstant(int *g_a, const int b) {
  int idx = blockIdx.x * blockDim.x + threadIdx.x;
  g_a[idx] += b;
}
\end{lstlisting}
