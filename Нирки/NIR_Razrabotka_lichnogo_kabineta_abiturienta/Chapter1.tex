\chapter{Изучение предметной области}

Информационная система — совокупность информационных, экономико-математических методов и моделей, технических, программных, технологических средств и специалистов, предназначенная для сбора, хранения, обработки и выдачи информации и принятия управленческих решений \cite{devinfsystem}.

Автоматическая система — система, состоящая из персонала и комплекса средств автоматизации его деятельности, реализующая информационную технологию установленных функций \cite{methdoinfsystem}.

Пользователь информационной системы — лицо, участвующее в функционировании информационной системы или использующее результаты ее функционирования \cite{infsystem}.

Основная задача этапа изучения предметной области — определение целей и задач на основе выявленных функций и информационных элементов автоматизируемого объекта \cite{exploitinfsystem}. 


\section{Сбор и обработка требований к информационной системе}

Для успешной разработки информационной системы необходимо осуществить сбор требований заказчика, провести обследование объекта и обосновать необходимость создания информационной системы \cite{highinfsystem}.

Требования к информационной системе «Личный кабинет абитуриента» основаны на документе: «Правила приема в ФГАОУ ВО «Волгоградский государственный университет» в 2021 году», № 01-23-1892.

Создание информационной системы обусловлено тем, что согласно пункту 4.6, одним из способов предоставить документы, необходимые для поступления, в ВолГУ, является сервис «Личный кабинет абитуриента», а также тем, что в связи с значительными изменениями в правилах приема в 2021 году, прошлая версия личного кабинета перестала соответствовать правилам приема.

Основываясь на правилах приема, выделены следующие основные требования к информационной системе:

\begin{enumerate} 
  \item реализация приема по программам бакалавриата, специалитета, и магистратуры;
  
  \item реализация приема по очной, очно-заочной, заочной формам обучения;
  
  \item реализация приема в пределах однопрофильного и многопрофильного конкурса;
  
  \item реализация приема за счет бюджетных ассигнований федерального бюджета, бюджетов субъектов Российской Федерации, местных бюджетов, по договорам об образовании, заключенным при приеме на обучение за счет средств физических и юридических лиц;
  
  \item реализация приема на места в пределах квоты приема на целевое обучение;
  
  \item реализация подачи заявление о приеме с приложением необходимых документов и согласием на обработку персональных данных. Заявление о приеме должно формироваться автоматически, на основе следующих сведений: фамилия, имя, отчество (при наличии), дата рождения, сведения о гражданстве (отсутствие гражданства), документ, удостоверяющий личность, сведения об образовании, реквизиты страхового свидетельства, от одной до десяти образовательных программ, на которые желает быть зачисленным, сведения о наличии или отсутствие особых прав (при наличии особых прав, прикрепить подтверждение наличие таких прав), сведения о сдачи ЕГЭ и его результатах, сведения о намерении участвовать в конкурсе по результатам общеобразовательных вступительных испытаний, проводимых ВолГУ самостоятельно, сведения о наличии или отсутствия у поступающего индивидуальных достижений, адрес электронной почты;
  
  \item реализация подачи согласия на зачисление. Согласие на зачисление должно формироваться автоматически на основе выбора направления для подачи согласия, а также сведений, предоставленных на этапе подачи заявления на поступление \cite{priembachelor}, \cite{priemspo}. 
\end{enumerate}

Результаты обследования представляют объективную основу для дальнейшего формирования технического задания.

\section{Разработка технического задания}

Техническое задание — документ, определяющий цели, требования и основные исходные данные, необходимые для разработки информационной системы, в соответствии с которым проводится разработка и ее приемка при вводе в действие \cite{gost34}.

При разработке технического задания необходимо решить следующие задачи:

\begin{enumerate} 
  \item установить общую цель создания информационной системы;
  
  \item установить общие требования к проектируемой системе;
  
  \item разработать требования, предъявляемые к информационному, программному, техническому, и технологическому обеспечению;
  
  \item определить состав подсистем и функциональных задач;
  
  \item определить этапы создания системы.
\end{enumerate}

\subsection{Общие сведения}

Полное наименование информационной системы: Личный кабинет абитуриента.

Заказчик: Федеральное государственное автономное образовательное учреждение высшего образования «Волгоградский государственный университет».

Разработчик: Отдел веб-технологий, Управление информатизации и телекоммуникаций, ФГАОУ ВО «Волгоградский государственный университет».

Плановые сроки начала работ: 01.06.2021.

Плановые сроки окончания работ: 15.07.2021.

Порядок оформления и предъявления заказчику результатов работ: работы по созданию информационный системы сдаются разработчиком поэтапно в соответствии с календарным планом проекта.

\subsection{Назначение и цели создания системы}

Назначение системы: Личный кабинет абитуриента   предназначен для обеспечения возможности подачи документов, необходимых для поступления, в электронной форме.

Цели создания системы: обеспечение сбора информации необходимой для формирования заявления на поступление и согласия на зачисление, автоматическое формирование заявлений на поступление и согласий на зачисление, обработка поданных заявлений и согласий сотрудниками приемной комиссии, отслеживание статусов поданных заявлений и согласий.

\subsection{Характеристика объектов автоматизации}

ФГАОУ ВО «Волгоградский государственный университет» — одно из крупнейших высших учебных заведений Волгоградской области. В структуру университета входят 9 институтов, около 50 специальных и общеуниверситетских кафедр. Постоянно обучается более 12 тысяч студентов и аспирантов. В университете реализуется 182 образовательные программы различных уровней.

Миссия ВолГУ – подготовка интеллектуальных кадров и производство новых знаний для экономики и социальной сферы Волгоградской области и Юга России по приоритетным направлениям развития науки, техники и технологий во взаимодействии с государством, бизнесом и общественностью на основе интеграции учебного процесса, фундаментальных и прикладных научных исследований.

Объектами автоматизации будут являться бизнес-процессы (таблица \ref{tab:processes}), выполняемые в приемной комиссии университета.

\begin{center}
\begin{longtable}{|p{3.6cm}|p{3.6cm}|p{3.6cm}|p{3.6cm}|}
\caption{Автоматизируемые бизнес-процессы}
\label{tab:processes}
\hline
Структурное подразделение & Наименование процесса & Возможность автоматизации & Решение об автоматизации в ходе проекта \\
\endfirsthead
\multicolumn{4}{l}{Продолжение таблицы \ref{tab:processes}} \\
\multicolumn{4}{l}{} \\
\hline
Структурное подразделение & Наименование процесса & Возможность автоматизации & Решение об автоматизации в ходе проекта \\
\endhead
\hline
Приемная комиссия&
Сбор информации необходимой для формирования заявления на поступление и согласие на зачисление&
Возможно&
Будет автоматизировано\\
\hline
Приемная комиссия&
Подача заявление на поступление&
Возможно частично&
Будет автоматизировано частично\\
\hline
Приемная комиссия&
Подача, удаление, отзыв согласий на зачисление&
Возможно частично&
Будет автоматизировано частично\\
\hline
Приемная комиссия&
Обработка поданных заявлений и согласий&
Возможно частично&
Будет автоматизировано частично\\
\hline
Приемная комиссия&
Оплата платных образовательных услуг&
Возможно&
Будет автоматизировано частично\\
\hline
\end{longtable}
\end{center}

\subsection{Требования к системе}

Требования к структуре и функционированию системы:

Система «Личный кабинет абитуриента» должна иметь микросервисную архитектуру.

В системе выделены следующие функциональных подсистемы:

\begin{enumerate} 
  \item подсистема, предназначенная для организации доступа пользователей к информационной система;
  
  \item подсистема сбора, обработки, предоставления данных;
  
  \item разработать требования, предъявляемые к информационному, программному, техническому, и технологическому обеспечению;
  
  \item подсистема хранения данных;
  
  \item подсистема формирования отчетности.
\end{enumerate}

Для организации информационного обмена между компонентами системы, а также организации доступа пользователей к системе должен использоваться протокол прикладного уровня HTTP и его расширение HTTPS.

Требования к численности и квалификации персонала системы и режиму его работы.

В состав персонала, необходимого для обеспечения эксплуатации системы в рамках соответствующих подразделений, необходимо выделение следующих ответственных лиц:

\begin{enumerate} 
  \item ответственный за проведение приемной кампании;
  
  \item ответственный за обработку заявлений и согласий по каждому институту на бюджетные и договорные места.
\end{enumerate}

К квалификации персонала, эксплуатирующего систему, предъявляются следующие требования: знание соответствующей предметной области, документов, регулирующих прием.

Требования к приспособляемости системы к изменениям.

Обеспечение приспособляемости системы должно выполняться за счет своевременного обновления системы под актуальные документы, регулирующие приемную кампанию.

Требования к защите информации от несанкционированного доступа:

\begin{enumerate} 
  \item реализация ролевой модели доступа;
  
  \item требование по наличию пароля для пользователей;
  
  \item требования по подтверждению пользователей посредством электронной почты;
  
  \item регистрация событий и действий пользователей;
  
  \item ограничение доступа к базам данных;
  
  \item назначение каждому работнику уникальной учетной записи.
\end{enumerate}

Перечень функций, задач подлежащей автоматизации:

Функция подачи документов на поступление.

Статусы заявления:

\begin{enumerate} 
  \item формируется — заявление заполняется пользователем и еще не отправлено на проверку сотруднику приемной комиссии;
  
  \item отправлено — заявление отправлено и ожидает проверки сотрудником приемной комиссии;
  
  \item на модерации — заявление проверяется сотрудником приемной комиссии;
  
  \item одобрено — заявление одобрено сотрудником приемной комиссии;
  
  \item отклонено — заявление отклонено сотрудником приемной комиссии.
\end{enumerate}

Система должна требовать ввод следующих данных:

\begin{enumerate} 
  \item фамилия;
  
  \item имя;
  
  \item отчество (при наличии);
  
  \item дата рождения;
  
  \item пол;
  
  \item СНИЛС;
  
  \item электронная почта;
  
  \item телефон;
  
  \item фотография 3\times4;
  
  \item серия паспорта;
  
  \item номер паспорта;
  
  \item кем выдан паспорт;
  
  \item дата выдачи паспорта;
  
  \item код подразделения;
  
  \item место рождения;
  
  \item разворот паспорта с фотографией;
  
  \item разворот паспорта с пропиской;
  
  \item адрес постоянной регистрации;
  
  \item адрес фактического проживания;
  
  \item уровень образования;
  
  \item тип документа;
  
  \item серия документа (при наличии);
  
  \item номер документа;
  
  \item наименование учебного заведения;
  
  \item дата выдачи;
  
  \item год окончания;
  
  \item скан-копия документа об образовании;
  
  \item скан-копия приложения с оценками;
  
  \item индивидуальные достижения (при наличии);
  
  \item целевые договоры (при наличии);
  
  \item олимпиады (при наличии);
  
  \item льготы (при наличии);
  
  \item выбранные направления подготовки;
  
  \item вступительные испытания;
  
  \item скан-копия согласия на обработку персональных данных;
  
  \item скан-копия согласия на распространение персональных данных;
  
  \item приложение к заявлению (при наличии);   
  
  \item иные документы (при наличии);
  
  \item заявление о приеме.
\end{enumerate}

Система должна формировать заявление на поступление на основе 1-34 пунктах. У пользователя системы должна быть возможность редактировать поданное заявление. При редактировании заявления, его статус сбрасывается на «формируется».

Функция подачи согласия на зачисление.

Абитуриент имеет возможность подать согласие на зачисление в случае, если его заявление на поступление получило статус «одобрено». Согласие на зачисление подается на одно из направление указанных в заявление на поступление. Согласие на зачисление автоматически формируется на основе данных, введенных на этапе подачи заявления на поступление и выбранного направления для зачисления.

Статусы согласия:

\begin{enumerate} 
  \item согласие отправлено — согласие отправлено и ожидает проверки сотрудником приемной комиссии;
  
  \item согласие на модерации — согласие проверяется сотрудником приемной комиссии;
  
  \item согласие одобрено — согласие одобрено сотрудником приемной комиссии;
  
  \item согласие отклонено — согласие отклонено сотрудником приемной комиссии;
  
  \item отзыв согласия отправлен — отзыв согласия отправлен и ожидает проверки сотрудником приемной комиссии;
  
  \item отзыв согласия на модерации — отзыв согласия проверяется сотрудником приемной комиссии;
  
  \item отзыв согласия одобрен — отзыв согласия одобрен сотрудником приемной комиссии;
  
  \item отзыв согласия отклонен — отзыв согласия отклонен сотрудником приемной комиссии.
\end{enumerate}

При отклонении согласия и отзыва согласия, сотрудник приемной комиссии должен написать в комментарии причину отклонения.

Функция обработки поданных заявлений и согласий.

Сотрудники приемной комиссии могут просматривать список всех согласий и заявлений согласно их роли. Согласия и заявление должны фильтроваться по уровню образовательной программы, институтам, статусам, сортироваться по дате изменения и уровню образовательной программы. Должен быть реализован нечеткий поиск согласий и заявлений по ФИО абитуриента.

При просмотре заявления, сотруднику приемной комиссии должна отображаться вся информация о заявлении. В панели обработки заявления должна быть реализована возможность одобрять и отклоняться заявления с указанием причины отклонения. Для синхронизации заявлений с системой учета, должен быть реализован функционал загрузки заявления в систему учета и выгрузки заявления из системы учета.

При просмотре согласия на зачисления, сотруднику приемной комиссии должна отображаться вся информация о согласии на зачисление. В панели обработки согласий должна быть реализована возможность одобрять и отклоняться заявления с указанием причины отклонения.

Функция оплаты платных образовательных услуг.

При наличии в учетной системе оформленных договоров на платные образовательные услуги, абитуриенту должна отображаться квитанция с qr-кодом для проведения оплаты.

\subsection{Состав и содержание работ по создания системы}

Состав и содержание работ по созданию информационной системы представлены на таблице \ref{tab:scopework}.
\begin{center}
\begin{longtable}{|p{3.9cm}|p{4.4cm}|p{3cm}|p{3.4cm}|}
\caption{Состав и содержание работ по созданию информационной системы}
\label{tab:scopework}
\hline
Этап & Содержание работ & Порядок приемки и документы & Ответственный \\
\endfirsthead
\multicolumn{4}{l}{Продолжение таблицы \ref{tab:scopework}} \\
\multicolumn{4}{l}{} \\
\hline
Этап & Содержание работ & Порядок приемки и документы & Ответственный \\
\endhead
\hline
Составление технического задания&
Разработка функциональных и нефункциональных требований к системе&
Утверждение технического задания&
Разработка — исполнитель, согласование — заказчик.\\
\hline
Техническое проектирование&
\begin{enumerate} 
  \item разработка сценариев работы системы;
  \item проектирование архитектуры веб-приложения;
  \item разработка макетов пользовательского интерфейса веб-приложения.
\end{enumerate}&
Утверждение дизайн-макета&
Разработка — исполнитель, согласование — заказчик.\\
\hline
Разработка программной части&
\begin{enumerate} 
  \item разработка модуля хранения данных и файлов;
  \item разработка REST-API для веб-приложения;
  \item разработка интеграции серверной части с системой учета;
  \item разработка одностраничного веб-приложения.
\end{enumerate}&
Приемка осуществляется в процессе испытаний&
Разработка — исполнитель, приемка — заказчик.\\
\hline
Предварительные автономные испытания&
\begin{enumerate} 
  \item проверка соответствия нефункциональным требованиям;
  \item проверка работоспособности системы в целом;
  \item проверка на соответствия всем функциональным требованиям. 
\end{enumerate}&
Приемка осуществляется в процессе испытаний&
Проведение испытаний и устранение недостатков — исполнитель, приемка — заказчик\\
\hline
Подготовка к опытной эксплуатации&
Разворачивание системы на промышленных серверах&
Приемка отсутствует&
Осуществление подготовки к опытной эксплуатации и устранение недостатков — исполнитель\\
\hline
Опытная эксплуатация&
\begin{enumerate} 
  \item эксплуатация с привлечением небольшого количества участников;
  \item доработки и повторы испытания до устранения недостатков.
\end{enumerate}&
Приемка отсутствует&
Устранение недостатков — исполнитель, проведение опытной эксплуатации — заказчик\\
\hline
Ввод в эксплуатацию&
Выполнение комплекса работ по подготовке программной части к вводу в эксплуатацию&
Приемка отсутствует&
Подготовка программной части — исполнитель, организация ввода в эксплуатацию — заказчик\\
\hline
Эксплуатация&
Промышленная эксплуатация&
Приемка отсутствует&
Эксплуатация — заказчик\\
\hline
\end{longtable}
\end{center}

\subsection{Порядок контроля и приемки системы}

Система подвергается испытаниям следующих видов:

\begin{enumerate} 
  \item предварительные автономные испытания — испытаниям подвергаются части системы по отдельности, если в составе имеется несколько подсистем или крупных модулей;
  
  \item опытная эксплуатация — эксплуатация на реальных данных, с реальными пользователями и с выполнением реальных задач.
\end{enumerate}

Состав, объем и методы предварительных испытаний определены в соответствующих этапах состава и содержания работ по созданию системы.

\subsection{Требования к составу и содержанию работ по подготовке объекта автоматизации к вводу системы в действие}

Для создания условия функционирования личного кабинета абитуриента, при которых гарантируется соответствие создаваемой системы требованиям, содержащимся в настоящем техническом задании, и возможность эффективного её использования, в организации должен быть проведены следующие работы:

\begin{enumerate} 
  \item подготовка персонала приемной комиссии к использованию информационной системы;
  
  \item разработка пользовательского соглашения, согласий на обработку и распространение персональных данных пользователей;
  
  \item развертывание системы на промышленных серверах;
  
  \item настройка системы доступа и создание необходимых учетных записей для сотрудников приемной комиссии;
  
  \item настройка интеграции с системой учета.
\end{enumerate}

\subsection{Требования к документированию}

Вся разрабатываемая документация должна быть выполнена на русском языке. Перечень документации приведен ниже (таблица \ref{tab:doc}).

\begin{table}[H]
\caption{Перечень документации}
\label{tab:doc}
\begin{center}
\begin{tabular}{|p{7.2cm}|p{7.2cm}|}
\hline
Этап & Перечень документов\\
\hline
Составление технического задания&
Техническое задание\\
\hline
Техническое проектирование&
Описание автоматизируемых функций\\
\hline
Разработка программной части&
Руководство пользователя\\
\hline
Ввод в действие&
Акт приемки системы в постоянную эксплуатацию\\
\hline
\end{tabular}
\end{center}
\end{table}

\subsection{Источники разработки}

Настоящее техническое задание разработано на основе следующих документов и информационных материалов:

\begin{enumerate} 
  \item ГОСТ 34.602-89 «Информационная технология. Комплекс стандартов на автоматизированные системы. Техническое задание на создание автоматизированной системы»;
  
  \item правила приема в ФГАОУ ВО «Волгоградский государственный университет» в 2021 году, № 01-23-1892;
  
  \item правила приема в ФГАОУ ВО ВолГУ на программы СПО в 2021 г., № 01-23-1805.
\end{enumerate}