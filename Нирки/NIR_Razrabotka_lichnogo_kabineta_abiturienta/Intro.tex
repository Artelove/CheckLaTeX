\intro
Автоматизированные информационные системы относятся к классу сложных систем, не только в связи с большой физической размерностью, а также из-за многозначности структурных отношений между их компонентами.

Актуальной задачей в информационном плане на текущий момент является обеспечение надежного управления всем объектом разнообразных данных, которые создаются, хранятся, и используются в различных информационных системах, в течении их жизненного цикла.

Разнообразие проблем, решаемых с помощью информационных систем, привело к появлению разнотипных систем, различающихся принципами построения и заложенными в них правилами обработки информации.

Объектом данного исследования явлеются информационная система «Личный кабинет абитуриента». Предметом исследования являются особенности, методы и средства разработки информационных систем.

В ходе данного научного исследования будут рассмотрены этапы создания, методы и средства разработки информационных систем, а в качестве практической части, будет разработана информационная система «Личный кабинет абитуриента».

Общая цель работы — исследовать этапы, методы и средства разработки информационных систем, разработать информационную систему «Личный кабинет абитуриента» с помощью изученных в результате исследования методов и средств. В ходе достижения общей цели научно-исследовательской работы решаются следующие задачи:


\begin{enumerate} 
  \item изучение предметной области, а именно сбор и обработка требований к информационной система, исследования и анализ конкурентных информационных систем, разработка технического задания;
  
  \item  техническое проектирование информационной системы «Личный кабинет абитуриента», включающее такие этапы, как обоснование выбора программного обеспечения и технической инфраструктуры, проектирование пользовательского интерфейса;
  
  \item  разработка клиентской части информационной системы «Личный кабинет абитуриента», включающая такие этапы, как регистрация новых пользователей, авторизация, подача заявления на поступление, подача согласия на зачисление, реализация функционала обработки заявлений и согласий.
\end{enumerate}