\conclusion
В ходе данной работы были изучены теоретические основы, типы и задачи машинного обучения, было осуществленно ознакомление с языком программирования Python, библиотеками Keras, TensorFlow, Pandas и Matplotlib. Был осуществлен поиск наборов для обучения в системе организации конкурсов по исследованию данных. Изученная теория была применена на практике, а именно была спроектрирована архитектура модели нейронной сети для решения задачи регрессии. Была написана реализция этой сети на языке Python и впоследствии обучена на наборе данных, содержащем данные о структуре 250000 различных молекул. Была написана программа для определения энергии основного состояния молекул по их структуре. Также во время разработки, отладки и тестирования модели сети была изучена среда разработки PyCharm. Были произведены исследование и тестирование полученной программы, выявлены статистические характеристики качества работы программы. 

В результате выполнения работы сформированы следующие компетенции:

Был проведен поиск нужной информации по теме «Использование библиотек Keras и TensorFlow в задачах машинного обучения» и ее анализ, что соотвествует компетенции:

УК--1 Способен осуществлять поиск, критический анализ и синтез информации, применять системный подход для решения поставленных задач.

Происходило взаимодействие с научным руководителем в устной и письменной формах по вопросам, которые касались реализации нейронной сети для определения энергии основного состояния молекул, что соответсвует компетенции:

УК--4 Способен осуществлять деловую коммуникацию в устной и письменной формах на государственном языке Российской Федерации и иностранном(ых) языке(ах).

В ходе данной работы были изучены теоретические материалы и спроектирована, реализована и обучена нейронная сеть определяющая энергию основного состояния молекулы по ее структуре, что соответствует компетенции:

ПК--1. Способен проводить научно-исследовательские и опытно-конструкторские разработки.

Проектирование и реализация нейронной сети были осуществлены на языке программирования Python в среде PyCharm, а также был составлен отчет в системе \LaTeX и презентация по всей проделанной работе, что соответсвует компетенции: 

ПК--2. Способен использовать методы компьютерного моделирования, современные методы обработки информации и оформлять полученные результаты в виде отчетов, презентаций и докладов.