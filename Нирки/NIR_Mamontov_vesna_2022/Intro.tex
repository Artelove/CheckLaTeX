\intro
Научно-исследователская работа посвящена использованию библиотек Keras и TensorFlow в задачах машинного обучения.

Целью данной работы является проектирование, реализация и обучение модели нейронной сети решающей задачу регрессии для нахождения энергии основного состояния квантово-механической системы. Для достижения этой цели необходимо выполнить следующие задачи:

\begin{enumerate}
    \item [1)] изучение теоретических материалов, посвященных разработке и обучению моделей нейронных сетей решающий задачу регрессии;
    \item [2)] поиск данных для обучения;
    \item [3)] проектирование архитектуры нейронной сети;
    \item [4)] реализация на языке Python с помощью библиотек Keras и TensorFlow c последующим обучением;
    \item [5)] проведение тестов и состовление отчета.
\end{enumerate}



В современных реалиах машинное обучение позволяет решать обширный
класс задач, которые практически невозможно решить стандартными методами
программирования. Эти задачи требуют сложных правил, которые трудно составить вручную, поэтому процесс составления правил становится задачей машины. Моделирование молекулярных свойств требует больших вычислительных затрат. Используя методы машинного обучения, можно создать модель, которая будет предсказывать молекулярные свойства не требуя больших вычислений.
