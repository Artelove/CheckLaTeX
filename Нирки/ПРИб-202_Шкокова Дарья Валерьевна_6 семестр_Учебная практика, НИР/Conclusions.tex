\conclusion
В рамках выполнения работы был проведен поиск и изучены современные иностранные и отечественные публикации в  журналах, а также учебники по данной тематике на русском и английском языках, средства для разработки программы, выполняющей создание математической модели с переменной структурой в среде визуального программировая, в результате чего была реализована программа.

При выполнение данной работы были получены следующие результаты:
\begin{enumerate}
\item[—] изучена необходимая литература по предметной области;
\item[—] изучены особенности систем с переменной структурой;
\item[—] изучены особенности перевернутого маятника с нелинейнным регулятором;
\item[—] реализована программа для создания математической модели перевернутого маятника с нелинейнным регутятором в системе с переменной структурой в среде визуального программирования;
\item[—] написан отчет по производственной практике;
\item[—] составлена презентация по отчету для производственной практики.
\end{enumerate}

В процессе прохождения производственной практики были освоены все компетенции, предусмотренные учебным планом.

УК-1. Способен осуществлять поиск, критический анализ и синтез информации, применять системный подход для решения поставленных задач. Компетенция была сформирована в процессе изучения темы системы с переменной структурой и подбора научной литературы для написания отчета по производственной практике (см. Глава 2).

УК-4. Способен осуществлять деловую коммуникацию в
устной и письменной формах на государственном языке Российской Федерации и иностранном(ых) языке(ах). Компетенция была сформирована в процессе написания отчета по производственной практике и при взаимодействии с ответственным за организацию практики.

ПК-1. Способен проводить научно-исследовательские и
опытно-конструкторские разработки. Компетенция была освоена в процессе разработки программы, моделирующей систему с переменной стуктурой с использованием блочных графических элементов в среде разработки MatLab Simulink (см. Глава 3).

