\chapter{Основные теоретические понятия о системах с переменной структурой}

Система с переменной структурой -- это класс динамических систем, которые могут изменять свою структуру в зависимости от различных факторов. В системе с переменной структурой структура системы может меняться в течение времени, что позволяет ей адаптироваться к различным условиям работы. Основными понятиями~\cite{4} такой структуры является:
\begin{enumerate}
\item[—] переменная структура;
\item[—] подсистемы;
\item[—] методы переменных структур;
\item[—] инвариантность;
\item[—] коммутационные уравнения;
\item[—] синтез управления систем с переменной структурой;
\item[—] анализ устойчивости систем с переменной структурой.
\end{enumerate}

Эти концепции и методы позволяют создавать более гибкие и адаптивные системы, которые могут приспосабливаться к изменяющимся условиям работы.

\section{Понятие о переменной структуре}
Переменная структура~\cite{5} -- возможность изменения структуры системы в течение времени, что позволяет ей адаптироваться к различным условиям работы. 

Примером переменной структуры является система управления двигателем автомобиля, где структура системы может меняться в зависимости от режима работы двигателя, такого как холостой ход, ускорение, торможение. В каждом из этих режимов работы система может использовать различные управляющие алгоритмы и параметры, чтобы обеспечить оптимальную работу двигателя. 

Другим примером является система управления роботом, где структура системы может меняться в зависимости от обнаруженных препятствий, изменения в окружающей среде, изменения задачи и т.д. В каждой из этих ситуаций система может использовать различные алгоритмы управления и параметры, чтобы обеспечить оптимальную работу робота.

Переменная структура является важной концепцией в СПС, так как она позволяет системе адаптироваться к изменяющимся условиям работы и повышать ее гибкость и эффективность.

\section{Подсистемы в переменной структуре}

Подсистемы -- набор подсистем, каждая из которых соответствует определенному режиму работы системы. В системе с переменной структурой можно выделить несколько различных режимов работы, и каждый режим соответствует своей подсистеме~\cite{6}. 

Каждая подсистема включает в себя управляющий алгоритм и набор параметров, которые оптимизированы для работы в определенном режиме. Например, в системе управления двигателем автомобиля можно выделить несколько подсистем, соответствующих различным режимам работы двигателя, таким как холостой ход, ускорение, торможение и т.д. В каждой подсистеме используются различные алгоритмы управления и параметры, чтобы обеспечить оптимальную работу двигателя в данном режиме. 

При переключении между подсистемами в системе с переменной структурой происходит изменение структуры системы, что позволяет ей адаптироваться к изменяющимся условиям работы. Например, в системе управления двигателем автомобиля переключение между подсистемами происходит автоматически в зависимости от текущего режима работы двигателя.

Использование подсистем в переменной структуре позволяет более эффективно управлять системами с переменной структурой, так как они позволяют переключаться между различными режимами работы системы в зависимости от ее текущего состояния и входных сигналов. Они также позволяют управлять сложными системами с переменной структурой более точно и эффективно.

\section{Методы систем с переменной структурой}

Методы переменных структур ~\cite{7} -- подход к моделированию и управлению систем с переменной структурой, основанный на переключении между различными режимами работы системы в зависимости от ее текущего состояния и входных сигналов. Основные методы систем с переменной структурой включают:

 \begin{enumerate}
\item[—] метод управления синхронизацией~\cite{8} -- метод, который использует нелинейные управляющие законы для обеспечения точного управления системой во всех режимах работы;
\item[—] метод управления с обратной связью по выходу~\cite{9}  -- метод, который использует выходной сигнал системы и нелинейную функцию для вычисления управляющего действия;
\item[—] метод управления с обратной связью по состоянию~\cite{10} -- метод, который использует линейную комбинацию состояний системы и нелинейную функцию для вычисления управляющего действия;
\item[—] метод управления с коммутацией~\cite{11} -- метод, который переключает между различными режимами работы системы в зависимости от ее текущего состояния и входных сигналов;
\item[—] метод динамической линеаризации~\cite{12} -- метод, который линеаризует систему в каждом режиме работы, чтобы использовать стандартные методы управления;
\item[—] метод дифференциальных уравнений с частными производными~\cite{13} -- метод, который использует дифференциальные уравнения с частными производными для описания динамики системы в различных режимах работы.
\end{enumerate}

Методы систем с переменной структурой позволяют более эффективно управлять системами с переменной структурой, так как они позволяют переключаться между различными режимами работы системы в зависимости от ее текущего состояния и входных сигналов. Они также позволяют управлять сложными системами с переменной структурой более точно и эффективно.

\section{Инвариантность в системе с переменной структурой}

 Инвариантность~\cite{14} -- свойство систем с переменной структурой, которое означает, что некоторые характеристики системы (например, устойчивость) сохраняются при изменении ее структуры. Инвариантность является важным свойством системы с переменной структурой, так как она позволяет сохранять определенные свойства системы при переключении между различными режимами работы и изменении структуры системы.

Например, в системе управления двигателем автомобиля инвариантность может быть связана с сохранением устойчивости двигателя в различных режимах работы. Другим примером может быть система управления роботом, где инвариантность может быть связана с сохранением точности и скорости движения робота при изменении окружающей среды или задачи.

Для обеспечения инвариантности в системе с переменной структурой можно использовать различные методы и алгоритмы управления, которые позволяют сохранять определенные характеристики системы при изменении ее структуры. Например, в методах систем с переменной структурой используются коммутационные уравнения, которые описывают переключение между различными подсистемами в системе с переменной структурой и обеспечивают сохранение определенных свойств системы при переключении между различными режимами работы.

Инвариантность является важным свойством системы с переменной структурой, которое обеспечивает ее стабильную работу и позволяет ей адаптироваться к изменяющимся условиям работы. Она также позволяет использовать систему с переменной структурой в различных приложениях, где требуется сохранение определенных характеристик системы при изменении ее структуры.

\section{Коммутационное уравнение в системе с переменной структурой} 
  Коммутационные уравнения ~\cite{15} -- уравнения, которые описывают переключение между различными подсистемами в систему с переменной структурой. Такая структура системы может меняться в зависимости от текущего состояния системы и входных сигналов, и переключение между различными подсистемами происходит автоматически в соответствии с заданными правилами. Коммутационное уравнение описывает этот процесс переключения.

Коммутационное уравнение обычно записывается в виде дифференциального уравнения и определяет изменение структуры системы в зависимости от текущего состояния системы и входных сигналов. Кроме того, коммутационное уравнение может быть использовано для определения оптимальных правил переключения между различными подсистемами, которые обеспечивают наилучшую работу системы в различных режимах.

Например, в системе управления двигателем автомобиля коммутационное уравнение может определять правила переключения между подсистемами, соответствующими различным режимам работы двигателя, таким как холостой ход, ускорение, торможение и т.д. В каждой из этих подсистем используются различные алгоритмы управления и параметры, чтобы обеспечить оптимальную работу двигателя в данном режиме.

Использование коммутационных уравнений в системах с переменной структурой позволяет обеспечить эффективное переключение между различными подсистемами в зависимости от текущего состояния системы и входных сигналов. Они также позволяют управлять сложными системами с переменной структурой более точно и эффективно.


\section{Синтез управления систем с переменной структурой} 

  Синтез управления систем с переменной структурой~\cite{4412471} -- процесс создания алгоритмов управления, которые обеспечивают стабильную работу систем с переменной структурой в различных условиях. Синтез управления в системе с переменной структурой является сложным и многогранным процессом, который включает в себя несколько этапов.

Основные этапы синтеза управления в системах с переменной структурой включают:
\begin{enumerate}
\item[—] определение модели системы: на этом этапе определяется математическая модель системы с переменной структурой, которая описывает ее динамику в различных режимах работы;
\item[—] выбор подхода к управлению: на этом этапе выбирается подход к управлению системой с переменной структурой, такой как методы систем с переменной структурой, методы оптимального управления, методы адаптивного управления и т.д.;
\item[—] проектирование управляющего алгоритма: на этом этапе проектируется управляющий алгоритм, который определяет правила переключения между различными подсистемами в системе с переменной структурой и оптимальные управляющие сигналы для каждой из подсистем;
\item[—] оптимизация управления: на этом этапе управляющий алгоритм оптимизируется для достижения заданных целей управления, таких как устойчивость, точность управления, быстродействие и т.д.;
\item[—] анализ управления: на этом этапе проводится анализ работы системы с переменной структурой с использованием различных критериев оценки, таких как критерии устойчивости, критерии точности, критерии быстродействия и т.д.;
\item[—] тестирование и настройка управления: на этом этапе производится тестирование управляющего алгоритма на реальной системе и его настройка для достижения оптимальной работы.
\end{enumerate}

Синтез управления в системах с переменной структурой позволяет обеспечить эффективное управление системами с переменной структурой, которые могут адаптироваться к изменяющимся условиям работы и работать в различных режимах. Он также позволяет управлять сложными системами с переменной структурой более точно и эффективно.

\section{Анализ устойчивости систем с переменной структурой} 


Анализ устойчивости систем с переменной структурой -- процесс оценки устойчивости систем с переменной структурой при различных условиях работы и моделирования ее поведения в ответ на внешние возмущения. Устойчивость системы с переменной структурой зависит от правил переключения между различными подсистемами и от параметров управления в каждой из них.

Для анализа устойчивости систем с переменной структурой используются различные методы и критерии. Наиболее распространенными методами являются методы Ляпунова и методы частотной области.

Методы Ляпунова основаны на построении функции Ляпунова, которая позволяет оценить устойчивость системы в зависимости от ее текущего состояния. Функция Ляпунова должна удовлетворять определенным условиям, чтобы гарантировать устойчивость системы. Если функция Ляпунова удовлетворяет этим условиям, то система считается устойчивой.

Методы частотной области основаны на анализе спектра системы и определении ее частотной характеристики. Частотная характеристика позволяет определить, насколько система устойчива в зависимости от ее частотных характеристик и параметров управления.

Кроме того, для анализа устойчивости систем с переменной структурой могут использоваться различные критерии, такие как критерии Рауса, критерии Михайлова и т.д. Критерии устойчивости позволяют определить, насколько система устойчива в зависимости от ее параметров и структуры.

Анализ устойчивости является важным этапом проектирования систем с переменной структурой, так как он позволяет определить, насколько система будет устойчива в различных режимах работы. Это позволяет проектировать более эффективные и надежные системы с переменной структурой, которые могут адаптироваться к изменяющимся условиям работы.
  