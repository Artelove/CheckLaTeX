\intro
 Одними из широко применяемых динамических систем яввляются системы с перемееной структурой~\cite{578556}. Данной системе характерно изменение своего поведения в зависимости от изменения многих факторов, которые воздействуют на неё, как в ней самой так и из вне. Моделирование систем с переменной структурой часто применяетсмя в таких областях как автоматическое управление~\cite{5498611}, робототехника~\cite{1}, автомобильной промышленности~\cite{2}, электроника~\cite{3}  и многое другое. Кроме того, с развитием технологий и появлением новых приложений, таких как автономные системы, робототехника и многие другие, моделирование систем с переменной структурой остается актуальной темой и требует дальнейших исследований и разработок.

 Целью работы является изучение различных систем с переменной структурой и разработка программы  на примере исследования какой-либо известной математической модели.

 В ходе выполнения производственной практики были выполнены следующие задачи:
\begin{enumerate}
\item[—] анализ необходимой литературы по предметной области;
\item[—] изучение особенностей систем с переменной структурой;
\item[—] изучение особенностей перевернутого маятника и нелинейнного регулятора;
\item[—] реализация программы для создания математической модели в среде визуального программирования;
\item[—] написание отчета по производственной практике.
\end{enumerate}

Отчет по производственной практике состоит из введения, трёх глав, заключения
и приложения.

Во введении обоснована актуальность выбранной темы, сформулированы цели и задачи изучения моделирования систем с переменной структурой.

В первой главе рассмотрено и описано понятие понятие динамических систем и их классификация, которая способствуют для понимания систем с переменной структурой. Описаны нелинейные элементы и характеристики.

Во второй главе рассмотрена и описана система с переменной структурой.

В третьей главе представлена реализация создания математической модели для перевернутого маятника и нелинейного регулятора в среде визуального программирования.

В заключении подведены основные итоги проделанной работы, а так же
сформулированы основные выводы.

В приложении предоставлен исходный код разработанной программы.

