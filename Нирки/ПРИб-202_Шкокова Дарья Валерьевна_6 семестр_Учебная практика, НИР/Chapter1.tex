\chapter{Основные определения и особенности динамических систем}

Общие закономерности, обладающие колебательными процессами в системах различной физической природы, являются основой науки, которая называется теорией колебаний. Под колебательным процессом понимают то, что связано с фактом определенного движения в исследуемой системе, или то, что связано с процессом перехода от одного движения к другому. Под определенным движением подразумевается повторяемость и определенная устойчивость. Переходные процессы характеризуются тем определенным движением, к которому они подступают. Множество таких переходных  процессов определенного движения образует его область притяжения. Изменения определенного движения, которые происходят в результате изменений какого-нибудь физического параметра исследуемой системы при его переходе через некоторое значение, называется бифуркацией. Если при этом изменение определенного движения происходит достаточно быстро, то говорят о <<жестком>> возникновении нового режима. Иначе называют <<мягким>>. Колебательные явления, возникающие в так называемых нелинейных системах, называются нелинейными колебаниям. Рассмотрим общий класс систем - динамические системы.

\section{Понятие динамической системы}

В общепринятом понимании под динамической системой можно представить некоторое явления, развивающееся во времени или во времени и пространстве по некоторому обусловленному закону~\cite{Motova}. Все, что может приводиться в движение, прогрессировать, усовершенствоваться или деградировать и при этом формировать какой-то закон трактуется как динамическая система. Задачей природоведения является изучение количественного предсказания развития системы во времени и пространстве, что формирует собой некую математическую задачу. Эта задача определяется и решается в рамках динамической системы.

 
В теории динамических систем под такой системой понимается математическая модель исследуемой физической динамической системы. Если определено понятие систояния системы и введен оператор развития, позволяющий однозначно установить соответствие между начальным состоянием  и состоянием в любой последующий момент  времени, то такую модель можно назвать явно заданной. Оператор развития может быть задан с помощью обыкновенных дифференциальных, интегральных или уравнений, дискретных отображений, а также в формате матриц, графов и т.д.~\cite{6389954}.

Векторной формой динамической системы является вид:
\begin{equation}
\dot x= F(x),
\end{equation}

\noindent где $F(x)$ - вектор функции размерности $N$.

Для рассмотрения динамической системы мы можем использовать два простых способа, которые опишут поведение данной системы со вмеми разнообразиями. 

Первый способ, если рассмотривать ее со стороны оепратора развития $T$ и фазового пространства $\Phi$~\cite{Tihomirov}. Данные характеристики дают чёткое определение того, как система ведет себя в данном пространстве. Они дают нам понять как выглядет состояние $x$ её в данный момент времени $t$. Опираясь на состояние и времени, от которого оно зависит, мы можем вычислить дальнейшие состояния в последующие моменты времени динамической системы $x((t+\Delta t)$.  Изменению состояния $x_i$ отвечает в фазовом пространстве движение соответствующей точки, которая называется изображающей. При этом движении изображающая точка описывает кривую, называемую фазовой траекторией. Фазовое пространство и оператор $T$ составляют математическую модель динамической системы. Исследование поведения динамической системы при таком подходе сводится к изучению характера разбиения фазового пространства на траектории и к выяснению зависимости структуры этого разбиения от значений физических параметров системы. 

Другой способ, применяемы к изучению динамических систем основан на рассмотрении функциональной стороны исследуемой системы. Данный подход может трактоваться невозможностью или отсутствием проникнуть во все тонкости внутренней структуры динамической системы. Поэтому система в этот случае характеризуется как некий <<черный ящик>>, обладающий некоторыми входными и выходными параметрами. Между этими параметрами реализуется связь, определяемая некоторым оператором. Таким образом, математическая модель при втором подходе определяется пространствами входных параметров и выходных. 

\section{Классификация динамических систем}

Классификация динамической системы может определяться по нескольким параметрам, например, по закону эволюции, фазовому пространству или наличию потерь и поступлению энергии.

Динамические системы можно классифицировать в зависимости от изменения оператора эволюции $T$. Если данный оператор $T$ соответствует принципу суперпозиции, то динамическая система является линейной, иначе система является нелинейной. Если состояние динамической системы и закон эволюции заданы в любой момент времени, то считается, что система с непрерывным временем. Если состояние системы заданы только в отдельные промежутки времени, то система с декретным временем. 

Динамические системы можно классифицировать по поведению фазового пространства $\Phi$. Состояние системы задается набором некоторых величин $x_i$, $i=1,2,...,N$, или функций $x_i \bold{(r)}$, $\bold r \in \bold{R}^M$. Величины $x_i$ называются динамическими переменными, связанные с рассматриваемыми количественными характеристиками динамической системы и в реальных системах могут быть изменены. Если $x_i$ являются переменным и их число конечно, то фазовое пространство $\Phi$ системы представляет собой арифметическое пространство. Динамические системы с таким пространством называются системами с сосредоточенными параметрами. Данные системы описываются обыкновенными дифференциальными уравнениями или отображением конечной последовательности. Существуют также системы с бесконечным фазовым пространством. Их состояние $x_i$ определяется функцией $x_i (r_1, r_2,...)$ некоторых переменных $r_k$, то фазовое пространство $\Phi$ является функциональным и его размерность бесконечна. В общем случае переменные $r_k$ являются некими координатами. Они представляются в таком виде, если параметры непрерывно распределены в пространстве с какой-то плотностью и зависят от пространственных координат. Такие системы принято называть распределёнными системами. Они описываются дифференциальными уравнениями в частных производных или интегральные уравнениями ~\cite{8604443}. 

Динамические системы также можно распределить на две большие группы: консервативные и неконсервативные системы. Консервативными системами являются системы, в которых нет потерь и покупательной энергии. В свою очередь неконсервативные системы имеют данные параметры. 

\section{Нелинейные элементы и нелинейные характеристики}

Система будет считать линейной, если в ней не возникает никаких хаотичных колебаний и действует принцип суперпозиции. Например, если взять два допустимых движения некоторой системы, обозначим их $x_1(t)$ и $x_2(t)$, то она считается линейной, потому что $c_1 x_1(t)+c_2 x_2(t)$ тоже будет допустимым движением. А сам принцип суперпозиции запишем в математическом виде:
\begin{equation}
L[x_i] = f(t),
\end{equation}

\noindentгде $x_i=(x_1, x_2,...)$ - независимые динамические переменные. Теперь если на вход пустить две разные функции $f_1 (t)$ и $f_2 (t)$ с соответствующими откликами $x_1(t)$ и $x_2(t)$, поэтому если система линейна, то уравнение будет выглядеть
\begin{equation}
L[c_1 x_1 + c_2 x_2] = c_1 f_1 (t) + c_2 f_2(t).
\end{equation}

Существование такого уравнения возможно, только если дифференциальное уравнение содержит только первые степени, поэтому в нелинейных системах неизвестные функции или комбинации для их производной, имеющие степень больше первой. 

Нелинейность в механических и электромагнитных системах может возникать из-за:
\begin{itemize}
\item [-] зависимости напряжения от деформации;
\item [-] зависимости напряжения от тока;
\item [-] нелинейного ускорения или кинетических эффектов;
\item [-] нелинейной массовой силы;
\item [-] геометрической нелинейности.
\end{itemize}

\noindent Все это приводит систему к нелинейному виду.










