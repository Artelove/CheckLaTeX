\conclusion
Основной целью данной работы является разработка приложения для упрощенного использования различных методов интерполяции пространственных данных для построения качественных цифровых моделей местности.

Для достижения цели были выполнены следующие задачи:
\begin{enumerate}
    \item [1)] изучение существующих методов интерполяции для построения цифровых моделей местности;
    \item [2)] реализация выбранных методов интерполяции;
    \item [3)] проектирование приложения для упрощенного использования методов интерполяции;
    \item [4)] агрегация методов интерполяции в рамках одного приложения;
    \item [5)] тестирование и анализ полученных результатов.
\end{enumerate}

Разработанное приложение позволяет пользователям быстро и удобно создавать цифровые модели местности с использованием различных методов интерполяции.

В процессе работы были выявлены некоторые ограничения и возможности для дальнейшего улучшения приложения. Например, можно рассмотреть возможность добавления новых методов интерполяции, улучшения пользовательского интерфейса и оптимизации производительности.

В результате выполнения работы были сформированы следующие компетенции:

При выполнении работы был проведен поиск нужной информации по теме «Разработка программного обеспечения для построения цифровых моделей местности» и ее анализ, что соответствует компетенции:

УК-1. Способен осуществлять поиск, критический анализ и синтез информации, применять системный подход для решения поставленных задач.

Происходило взаимодействие с научным руководителем в устной и письменной формах по вопросам, которые касались реализации этапов построения матриц с использованием методов интерполяции обратно взвешенных расстояний, сплайнами и радиально-базисной функцией, что соответсвует компетенции:

УК-4. Способен осуществлять деловую коммуникацию в устной и письменной формах на государственном языке Российской Федерации и иностранном(ых) языке(ах).

В ходе данной работы были изучены теоретические материалы, реализованы методы интерполяции обратно взвешенных расстояний, сплайнами и радиально-базисной функцией для построения поверхности, также разработан интерфейс приложения и реализован весь функционал программного обеспечения:

ПК-1. Способен проводить научно-исследовательские и опытно-конструкторские разработки.

Все построения цифровых моделей местности были проведены непосредственно в реализованном приложении, реализованное программное обеспечение написано на языке программирования C\# с использованием модульной платформы для разработки ПО .NET, а также был составлен отчет в системе \LaTeX и презентация по всей проделанной работе, что соответсвует компетенции:

ПК-2. Способен использовать методы компьютерного моделирования, современные методы обработки информации и оформлять полученные результаты в виде отчетов, презентаций и докладов.

Были определены требования к функционалу программного обеспечения для построения цифровых моделей местности, также были простроены диаграмма вариантов использования и диаграмма классов, что соответсвует компетенции:

ПК-5. Способен анализировать требования к программному обеспечению и разрабатывать технические спецификации на программные компоненты.

Был проведен анализ документации и работы других систем(ГИС <<Панорама>>, QGIS, Arc-GIS) в области работы c инерполяцией поверхностных данных, были изучены используемые методы в данных системах и спроектированы собственные разработки, что соответствует компетенции:

ПК-6. Способен проводить концептуальное, функциональное и логическое проектирование информационных систем.

Были изучены и проанализированы варианты и методы построения интерфейсных основ на примере иных микросервисов с узконаправленными назначениями, что соответствует компетенции:

ПК-7. Способен проектировать пользовательские интерфейсы \textcolor{red}{ по готовому образцу или концепции интерфейса}.

