\chapter{Листинг разработанной программы для построения цифровых моделей местности с использованием реализованных методов интерполяции}

\begin{lstlisting}[caption={Метод интерполяции сплайном}, label={ls:a:01}]
static double[,] calculate_spline()
{
 double[,] grid = new double[N, M];
 double[,] a = new double[Number_points + 3, Number_points + 3];
 double[,] a1 = new double[Number_points + 3, Number_points + 3];
 double[] b = new double[Number_points + 3];
 double[] x = new double[Number_points + 3];
 for (int i=0; i < Number_points + 3; i++)
 {
    x[i] = 0;
    if (i < Number_points)
    {
        b[i] = _Z[i];
    }
    else
    {
        b[i] = 0;
    }
    for (int j=0; j < Number_points + 3; j++)
    {
        a1[i,j] = 0;
    }
 }
 for (int i=0; i < Number_points + 3; i++)
 {
    for (int j=0; j < Number_points + 3; j++)
    {
        if(i < Number_points)
        {
            if (j < Number_points)
            {
                if (i!=j)
                {
                    a[i, j] = ((_X[i] - _X[j]) * (_X[i] - _X[j]) + (_Y[i] - _Y[j]) * (_Y[i] - _Y[j]))
                             * Math.Log((_X[i] - _X[j]) * (_X[i] - _X[j]) + (_Y[i] - _Y[j]) * (_Y[i] - _Y[j]));
                }
                else
                {
                    a[i, j] = 0;
                }
            }
            else
            {
                a[i, Number_points] = _X[i];
                a[i, Number_points + 1] = _Y[i];
                a[i, Number_points + 2] = 1;
            }
        }
        else
        {
            if (j < Number_points)
            {
                a[Number_points, j] = 1;
                a[Number_points + 1, j] = _X[j];
                a[Number_points + 2, j] = _Y[j];
            }
            else
            {
                a[i, Number_points] = 0;
                a[i, Number_points + 1] = 0;
                a[i, Number_points + 2] = 0;
            }
        }
    }
 }
 Gauss(a, a1, b, x, Number_points + 3);
 for (int i = 0; i < N; i++)
 {
    for (int j = 0; j < M; j++)
    {
        grid[i, j] = Spline(x,i,j);
    }
 }
 for (int i = 0; i < N; i++)
 {
    for (int j = 0; j < M; j++)
    {
        if (MATRIX[i, j] != 0)
        {
            grid[i, j] = MATRIX[i, j];
        }
    }
 }
 return grid;
}
\end{lstlisting}

\begin{lstlisting}[caption={Метод интерполяции с помощью радиально-базисной функции}, label={ls:a:02}]
static double[,] calculate_RBF()
{
 double[,] grid = new double[N, M];
 double[,] a = new double[Number_points, Number_points];
 double[,] a1 = new double[Number_points, Number_points];
 double[] b = new double[Number_points];
 double[] x = new double[Number_points];
 for (int i = 0; i < Number_points; i++)
 {
    x[i] = 0;
    b[i] = _Z[i];
    for (int j = 0; j < Number_points; j++)
    {
        a1[i, j] = 0;
        if (i == j)
        {
            a[i, j] = 1;
        }
        else
        {
            a[i, j] = EvklidBasis(_X[j], _Y[j], _X[i], _Y[i]);
        }
    }
 }
 Gauss(a, a1, b, x, Number_points);
 for (int i = 0; i < N; i++)
 {
    for (int j = 0; j < M; j++)
    {
        grid[i, j] = RBF(x, i, j);
    }
 }
 for (int i = 0; i < N; i++)
 {
    for (int j = 0; j < M; j++)
    {
        if (MATRIX[i, j] != 0)
        {
            grid[i, j] = MATRIX[i, j];
        }
    }
 }
 return grid;
}
\end{lstlisting}