\chapter*{Ход выполнения практики}
\addcontentsline{toc}{chapter}{Ход выполнения практики}

\section*{Задания и отметки о выполнении}

\task{Написать обзор по предметной области исследования, основываясь на научной, учебной и учебно-методической литературе, как на русском, так
и на английском языках.}

\task{Изучить основные методы интерполяции пространственных данных для построения цифровых моделей.}

\task{Провести анализ требований к функционалу программы для построения поверхностной интерполяции цифровых моделей местности.}

\task{Спроектировать и реализовать приложение для построения цифровых моделей местности на языке С\#.}

\task{Протестировать и проанализировать полученные результаты.}

\task{Подготовить отчет по практике в соответствии с требованиями нормоконтроля.}

\task{Создать презентацию в соответствии с данным отчетом и устный доклад для публичной защиты.}

\clearpage
\section*{Отзыв о работе обучающегося}

В ходе выполнения производственной практики, преддипломной практики Ивина Д.К. изучила существующие методы интерполяции для построения цифровых моделей местности и реализовала выбранные методы. Дарья Константиновна спроектировала и реализовала программное обеспечение для упрощенного использования различных методов интерполяции пространственных данных для построения более качественных цифровых моделей местности. Подготовила отчет в соответствии с требованиями нормоконтроля в издательской системе \LaTeX. Успешно освоила компетенции, предусмотренные учебным планом. 

Ивина Дарья Константиновна полностью справилась с поставленными перед ней задачами. Замечания к работе студента отсутствуют. Считаю, что ее работа заслуживает оценки <<отлично>>.

\progressapprov
