\intro
Данная работа посвящена разработке программного обеспечения для построения цифровых моделей местности. Целью работы является разработка приложения для упрощенного использования различных методов интерполяции пространственных данных для построения более качественных цифровых моделей местности. 

Для достижения поставленной цели были определены следующие задачи: изучение существующих методов интерполяции для построения цифровых моделей местности, реализация выбранных методов интерполяции, проектирование приложения для упрощенного использования методов интерполяции, агрегация методов интерполяции в рамках одного приложения, тестирование и анализ полученных результатов.

Цифровые модели местности являются важным инструментом в геоинформационных системах, так как они представляют собой двумерное или трехмерное изображение поверхности земли, которое может быть использовано для анализа и визуализации различных географических данных. В геоинформационных системах они используются для создания карт высот, профилей местности, анализа склонов и наклонов, расчета объемов земляных работ, планирования маршрутов и многих других задач. Они также могут быть использованы для моделирования поверхности земли в различных масштабах, от мелких участков до целых регионов. 

Цифровые модели местности могут быть созданы с использованием различных методов, таких как лазерное сканирование, фотограмметрия, интерполяция и другие. Они могут быть представлены в различных форматах, таких как растровые и векторные данные, что позволяет использовать их в различных ГИС-приложениях.

Актуальность темы обусловлена необходимостью создания более точных и детальных цифровых моделей местности, которые могут быть использованы в различных отраслях. Кроме того, разработка программного обеспечения для построения цифровых моделей местности может существенно упростить работу специалистов в области геоинформатики и смежных областей.

В первой главе описаны основные понятия и классификация цифровых моделей, а именно: цифровых моделей местности, цифровых моделей рельефа, цифровой модели контуров и цифровой модели специально инженерного назначения. Также описаны области их применения и некоторые особенности, которые присущи определенным цифровым моделям.

Во второй главе подробно описаны выбранные методы интерполяции и их реализация, приведены части кода, отвечающие за определенные функции, такие как запись входных значений в матрицу из текстового файла, заполнение неизвестных значений, используя один из методов интерполяции, таких как интерполяция обратно взвешенных расстояний, интерполяция сплайнами, интерполяция с использованием радиально-базисной функции.

В третьей главе описаны этапы разработки приложения для построения цифровых моделей местности, приведены диаграммы классов и вариантов использования для большего понимания архитектуры программного обеспечения, также обусловлен выбор языка программирования и платформы для создания приложения. Подробно описан интерфейс и весь функционал программного обеспечения, ньюансы связанные с загрузкой текстового файла и пример простроенной цифровой модели местности с выбранными настройками из окна приложения.
