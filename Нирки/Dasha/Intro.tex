\intro

Данная работа посвящена рассмотрению методов построения цифровых моделей рельефа (ЦМР) и их сравнению. Целью работы является реализация трех методов построения ЦМР: метода обратно взвешенных расстояний, метода интерполяции сплайнами и метода с использованием радиально-базисной функции для построения матриц, таких как матрица залегания грунтовых вод, матрица коэффициента пористости и фильтрации. В качестве исходные данных были использованы данные геодезической съемки, результаты исследования почвы в скважинах и в пределах продольных профилей рек, расположенных вблизи села Ольховка Волгоградской области. 

В настоящее время основной частью экспериментальных и теоретических исследований в разных сферах науки является построение рельефа местности и его исследование по полученным моделям. Возможность прогнозировать природные явления, например цунами, быстроразвивающиеся бурные паводки, наводнение и т.д., доступна при помощи математико-картографических методов моделирования. Различные экологические показатели можно определять при помощи полученных данных.

Сфера применения ЦМР не ограничена одной областью, ее используют для глобальной классификации земельного покрова, точное составление карт и классификация земной поверхности в глобальном масштабе является очень важной предпосылкой для крупномасштабного моделирования геологических процессов, также ЦМР используют в экологии, гидрологии, геоморфологии и климатологии. Таким образом, использование цифровых моделей рельефа является необходимым условием для кодирования изображений снятых со спутника и коррекции эффектов местности. 

% Во <<Введении>>  требуется обосновать актуальность выбранной темы, которая определяется значимостью ее теоретического и практического решения. Формулируются цель и задачи практики, определяются объект и предмет исследования, методология исследования. Целесообразно также охарактеризовать степень разработанности темы в отечественной и зарубежной литературе, изложить структуру работы. Этот раздел должен включать в себя краткое содержание осовных разделов отчета. Введение не должно составлять более 8--10\% от общего объема отчета (3--5 страниц). Так как <<Введение>> содержит в себе оприсание постановки задачи и освещает то, как тема разработана ранее другими авторами, то этот раздел предполагает наличие ссылок на другие источники. \texttt{Приведем пример того}, как ссылки должны~быть оформлены~\cite{CitekeyArticle, Boreskov2010}.

