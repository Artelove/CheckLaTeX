\conclusion
В ходе выполнения данной учебной практики были изучены теоретические материалы, описывающие методы построения цифровых моделей рельефа и их классификацию.

Данные знания были применены на практике для написания программы для построения матриц с использованием методов интерполяции, таких как метод обратно взвешенных расстояний, метод интерполяции сплайнами, метод с использованием радиально-базисной функции на языке С\#. Для построенных матриц был проведен сравнительный анализ. Матрицы, построенные методом обратно взвешенных расстояний, больше подходят для построения поверхности с малым количеством исходных точек. Метод сплайн интерполяции дает более гладкие матрицы и лучше интерполирует поверхности с большим количеством опорных точек. Метод радиально-базисных функций является более гибким, так как имеет возможность менять радиально-базисную функцию под конкретную задачу и получать различные результаты, с возможностью выбора самого подходящего. Методы радиально-базисных функций и сплайн интерполяция требуют решения системы линейных уравнений методом Гаусса, что увеличивает их вычислительную сложность до $O(n^3)$, что отрицательно сказывается на производительности программы при большом количестве точек.

Было спроектировано приложение проводящее интерполяцию поверхностных данных с возможностью выбора метода интерполяции. Также были изучены теоретические основы используемых методов интерполяции, был проведен их анализ и построены алгоритмы численных методов для решения задач интерполяции функций двух переменных. 

В результате выполнения работы были сформированы следующие компетенции:

При выполнении работы был проведен поиск нужной информации по теме «Методика построения цифровых моделей рельефа» и ее анализ, что соответствует компетенции: 

УК-1. Способен осуществлять поиск, критический анализ и синтез
информации, применять системный подход для решения поставленных
задач.

Происходило взаимодействие с научным руководителем в устной и письменной формах по вопросам, которые касались реализации этапов построения матриц с использованием методов интерполяции обратно взвешенных расстояний, сплайнами и радиально-базисной функцией, что соответсвует компетенции:

УК-4. Способен осуществлять деловую коммуникацию в устной и
письменной формах на государственном языке Российской
Федерации и иностранном(ых) языке(ах).

Был проведен анализ требований к функционалу программы для поверхностной интерполяции матриц качеств, что соответсвует компетенции:

ПК-5. Способен анализировать требования к программному обеспечению
и разрабатывать технические спецификации на программные компоненты.

Был проведен анализ документации и работы других систем(ГИС <<Панорама>>, QGIS, Arc-GIS) в области работы c инерполяцией поверхностных данных, были изучены используемые методы в данных системах и спроектированы собственные разработки, что соответствует компетенции: 

ПК-6. Способен проводить концептуальное, функциональное и логическое
проектирование информационных систем.

Были изучены и проанализированы варианты и методы построения интерфейсных основ на примере иных микросервисов с узконаправленными назначениями, что соответствует компетенции:

ПК-7. Способен проектировать пользовательские интерфейсы по готовому
образцу или концепции интерфейса.
