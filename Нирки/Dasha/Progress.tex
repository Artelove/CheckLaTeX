\chapter*{Ход выполнения практики}
\addcontentsline{toc}{chapter}{Ход выполнения практики}

\section*{Задания и отметки о выполнении}

\task{Написать обзор по предметной области исследования,
основываясь на научной, учебной и учебно-методической литературе, как на русском, так
и на английском языках. }

\task{Изучить основные методы интерполяции данных для построения цифровых моделей.}

\task{Провести анализ требований к функционалу программы для выполнения поверхностной интерполяции матриц качеств.}

\task{Реализовать методы построения матриц высот и качеств с использованием методов интерполяции, таких
как метод обратно взвешенных расстояний, метод интерполяции сплайнами,
метод с использованием радиально-базисной функции на языке С\#.}

\task{Подготовить отчет по практике в соответствии с требованиями нормоконтроля.}

\task{Создать презентацию в соответствии с данным отчетом и устный доклад для публичной защиты.}

\clearpage
\section*{Отзыв о работе обучающегося}

В ходе выполнения учебной практики, проектно-технологической практики Ивина Д.К. изучила способы построения цифровых моделей рельефа и матриц качеств с использованием различных методов интерполяции. Дарья Константиновна реализовала методы построения цифровых матриц с использованием ограниченного объема исходных данных. Подготовила отчет в соответствии с требованиями нормоконтроля в издательской системе \LaTeX. Успешно освоила компетенции, предусмотренные учебным планом. 

Ивина Дарья Константиновна полностью справилась с поставленными перед ней задачами. Замечания к работе студента отсутствуют. Считаю, что ее работа заслуживает оценки <<отлично>>.

\progressapprov
