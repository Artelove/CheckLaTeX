\conclusion

В ходе данной научно-исследовательской работы было рассмотрено что такое блокчейн, на каких информационных технологиях строится его концепция, основные принципы технологии блокчейн, в каких областях применяют данную технологию, какую роль играет оракул в децентрализованных системах. Были подробно рассмотрены два основных алгоритма консенсуса в децентрализованных системах, их преимущества и недостатки. Проведено сравнение централизованных, распределенных и децентрализованных приложений. Исследована платформа Ethereum, как единая децентрализованная виртуальная машина для выполнения программ, называющихся смарт-контракты.

Для рассмотрения применения технологии блокчейн в распределенных системах на практике, был разработан протокол, для проведения тайных электронных голосований с использованием технологии блокчейн, разработанный протокол был реализован в виде смарт-контракта, написанного на языке Solidity, под платформу Ethereum. Проведено модульное тестирование разработанного смарт-контракта, для минимизирования возможных ошибок и имитации работы приложения, согласно разработанному протоколу.

В рамках учебной практики, научно-исследовательской работы были освоены следующие компетенции:

УК-1. Способен осуществлять поиск, критический анализ и синтез информации, применять системный подход для решения поставленных задач.

Компетенция была сформирована в процессе анализа методов обработки больших данных, а также поиска библиографии с помощью библиографических баз Scopus, Book.ru, ADS, Юрайт, Лань.

УК-4. Способен осуществлять деловую коммуникацию в устной и письменной формах на государственном языке Российской Федерации и иностранном(ых) языке(ах).

Компетенция была сформирована в процессе деловой коммуникации с научным консультантом в письменной и устной формах по вопросам, касающихся научно-исследовательской работы.

ПК-1. Способен проводить научно-исследовательские и опытно-конструкторские разработки.

Компетенция была сформирована в процессе исследования технологии блокчейн, платформы Ethereum, различных протоколов электронных голосований, разработки протокола проведения электронных голосований с использованием технологии блокчейн, программной реализации разработанного протокола в виде смарт-контракта, под платформу Ethereum.

ПК-2. Способен использовать методы компьютерного моделирования, современные методы обработки информации и оформлять полученные результаты в виде отчетов, презентаций и докладов.

Компетенция была сформирована в процессе использования среды для разработки децентрализованных приложений Hardhat, реализации алгоритма слепой подписи, разработки протокола проведения электронных голосований с использованием технологии блокчейн, оформления отчета по производственной практике, научно-исследовательской работы с использованием набора макрорасширений \LaTeX, подготовки мультимедийной презентации и доклада по результатам производственной практики, научно-исследовательской работы.