\conclusion
Основной целью данной работы являлось проектирование и создание функциональных возможностей программного обеспечения для статического анализа кода системы компьютерной верстки \LaTeX, выдающей в качестве результата информационные сообщения, указывающие на  моменты несоответствия паттернов оформления работы относительно проверяемых заданных норм с помощью конфигурационного файла. Для достижения цели были выполнены следующие задачи:
\begin{enumerate}    
    \item выявление анализируемых характеристик;    
    \item создание и описание модели конфигурационного файла;    
    \item описание алгоритма выполняемых функций парсинга файлов документа; 
    \item описание алгоритма выполняемых функций проверки паттернов оформления;    
    \item составление и описание хода работы программы.
\end{enumerate}

В ходе данной работы были изучены материалы, связанные со статическим анализом, включая цели, задачи, преимущества и недостатки, была рассмотрена концепция написания документов с помощью системы верстки \TeX, было осуществлено ознакомление с основными командами \LaTeX, их аргументами и опциями, созданы и описаны методы парсинга файлов документа и реализованы описанные методы тестирования на соответствие паттернам оформления работы 

Изученная теория применена на практике, а именно – создан и описан конфигурационный файл являющийся шаблоном, по которому необходимо выполнять статический анализ, а так же описаны механизмы его проведения. А так же выполнено описание хода работы и алгоритмов функций програмного обеспечения.

По итогам работы был сфомированн отчет в системе \LaTeX.
В результате выполнения работы были сформированы следующие компетенции: 


УК-1. Способен осуществлять поиск, критический анализ и синтез информации, применять системный подход для решения поставленных задач.
Компетенция была сформирована в процессе проведения поиска необходимой информации по теме \guillemotleft Разработка программного обеспечения для статического анализа кода системы компьютерной верстки TeX\guillemotright и ее анализа.


УК-4. Способен осуществлять деловую коммуникацию в устной и письменной формах на государственном языке Российской Федерации и иностранном(ых) языке(ах).
Компетенция была сформирована в процессе выполнения работы по генерации идеи и постановки задачи, а также выбора используемых технологий с научным руководителем.

ПК-1. Способен проводить научно-исследовательские и опытно-конструкторские разработки. 
Компетенция была сформирована в процессе анализа особенностей строения команд, выделения параметров и аргументов, разделение блоков текста и разработки на основе сформированного представления алгоритма парсинга  системы компьютерной верстки \LaTeX.

ПК-2. Способен использовать методы компьютерного моделирования, современные методы обработки информации и оформлять полученные результаты в виде отчетов, презентаций и докладов. 
Компетенция была сформирована в процессе документирования используемых методов парсинга документа и фукнций тестирования паттернов оформления, оформления отчета по производственной практике, преддипломной практике с использованием набора макрорасширений LATEX, подготовки мультимедийной презентации и доклада по результатам производственной практики, преддипломной практики.


ПК-5. Способен анализировать требования к программному обеспечению и разрабатывать технические спецификации на программные компоненты. 
Компетенция была сформирована в процессе оценки и анализа требований к исходным данным и программе статического анализа системы верстки \LaTeX \ и выбора подходящих технологии для ее реализации.



ПК-6. Способен проводить концептуальное, функциональное и логическое проектирование информационных систем. 
Компетенция была сформирована в процессе разработки архитектуры и алгоритмов выполнения парсинга документа и тестирования паттернов оформления в программе для статического анализа системы верстки \LaTeX.


ПК-7. Способен проектировать пользовательские интерфейсы по готовому образцу или концепции интерфейса. 
Компетенция была сформирована в процессе проектирования визуальных пользовательских элементов взаимодействия при работе с основными функциями программы.
