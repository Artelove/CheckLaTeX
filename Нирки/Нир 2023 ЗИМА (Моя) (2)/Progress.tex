\chapter*{Ход выполнения практики}
\addcontentsline{toc}{chapter}{Ход выполнения практики}

\section*{Задания и отметки о выполнении}

\task{Написать обзор по предметной области исследования,
основываясь на научной, учебной и учебно-методической литературе, как на русском, так
и на английском языках. Необходимо и спользовать современную литературу на английском
и русском языках по тематике разработка программного обеспечения для статического анализа кода системы компьютерной верстки TeX, поиск которой можно осуществлять по библиографическим базам
Scopus, WoS, elibrary, ResearchGate, ADS, ЭБС Лань и др.}

\task{Создать и описать модель конфигурационного файла, испольуемого для статического анализа документа.}

\task{Описать и реализовать алгоритмы функций парсинга файлов документа.}

\task{Описать и реализовать алгоритмов функций проверки соответствия характеристикам конфигурационного файла.}

\task{Составить и описать блок-схему работы созданной программы.}

\task{Подготовить отчет по практике в соответствии с требованиями нормоконтроля кафедры ИСКМ. Оформить отчет по производственной практике, преддипломной практике с использованием набора макрорасширений \LaTeX.}

\task{Подготовить мультимедийную презентацию и доклад по результатам производственной практике, преддипломной практике.}

\clearpage
\section*{Отзыв о работе обучающегося}

В ходе выполнения производственной практики, преддипломной практики, Карагичев А.В. изучил цели из задачи статического анализа кода, проанализировал методы используемые для анализа, добавил реализацию алгоритмов парсинга файлом проверяемого документа и алгоритмов проверки на соответствие паттернам оформления в созданный проект разработки программного обеспечения. 
Подготови очет в соответствии с требованиями номкотроля в издательстве системе \LaTeX. Успешно освоил компетенции, предусмотренные учебным планом.
Карагичев Александр Владимирович полностью справился с поставленными перд ним задачами. Замечания к работе студента осутствуют. Считаю, что его работа заслуживает оценки \guillemotleftотлично\guillemotright.

\progressapprov
