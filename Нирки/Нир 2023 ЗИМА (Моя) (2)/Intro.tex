\intro
Производственная практика, научно-исследовательская работа посвящена процессу разработки программного обеспечения для статического анализа кода системы компьютерной верстки \TeX. Целью работы является ознакомится с имеющейся информацией о целях и задачах статического анализа кода, выявить его преимущества и недостатки, проанализировать методы используемые для анализа, и, используя данную теоретическую базу подготовить, описать и реализовать проект разработки ПО.

 В данной работе описана целесообразность использования статического анализа кода, закономерность построения файлов \LaTeX \ и его основных функций, алгоритмы парсинга и проверки на соответствие паттернам оформления.  Данные знания легли в основу практической работы, где был проработан и реализован файл конфигурации статического анализа и функции проверки анализируемых характеристик документа.

Раннее обнаружение и устранение ошибок является важным аспектом при разработке программного обеспечения. Это помогает снизить риски и затраты на исправление ошибок, которые могут возникнуть в будущем, а также повысить качество и надежность программного обеспечения.
Такой подход позволяет выявить проблемы до того, как они станут критическими. Это дает возможность разработчикам быстро реагировать на проблемы и исправлять их, не дожидаясь, пока они приведут к сбою в работе системы. Кроме того, раннее обнаружение ошибок помогает предотвратить распространение проблем на другие компоненты системы и снизить риск возникновения более серьезных проблем в будущем.
В целом, раннее обнаружение и устранение ошибок помогает обеспечить стабильность и надежность программного обеспечения, что является критически важным для многих приложений и систем. В разработке ПО кроме модульного и функционального тестирования для повышения качества продукта могут применяться практики статического анализа кода, которые являются самым простым и эффективным способом предотвращения дефектов и выявления несоответствий исходного кода принятому стандарту оформления. Статический анализ можно рассматривать как автоматизированный процесс обзора кода. Инструменты статического анализа непрерывно обрабатывают исходные тексты программ и выдают программисту рекомендации обратить повышенное внимание на определенные участки кода.
