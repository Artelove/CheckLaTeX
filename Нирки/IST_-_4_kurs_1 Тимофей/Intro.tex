\intro

Сегодня сложно переоценить роль мобильного телефона в жизни человека. С помощью этого компактного устройства мы в любой момент можем быстро связаться с людьми в любой точке мира, чтобы узнать интересующую нас информацию.

Android -- это самая популярная мобильная операционная система в мире, поскольку она доступна на огромном количестве устройств.
Раньше разработчикам приходилось писать разные версии программ не только под каждого производителя, но и под каждую модель. Благодаря существующим операционным системам для мобильных устройств, разработчику достаточно написать всего лишь одно приложение для определенной ОС, и оно будет работать на всех моделях смартфонов, работающих под управлением этой ОС \cite{book:1}.

В ходе данного научного исследования будут рассмотрены передовые технологии разработки Android использующие Kotlin, позволяющее создавать удобные и отзывчивые приложения, популярные паттерны проектирования, а также будет создаано мобильное приложение.

Основной целью данной работы является изучение архитектур и базовых принципов разработки, проектирование и реализация мобильного приложения для диабетиков DiabeticDiary на языке программирования Kotlin, а также освоение компетенций в соответствии с учебным планом.

Для достижения цели необходимо выполнить следующие задачи:
\begin{enumerate}
    \item изучение литературы по предметной области;
    \item изучение современных технологий мобильной разработки;
    \item изучение архитектурных решений в мобильной разработке;
    \item работа с локальными данными мобильного устройства;
    \item проектирование и разработка мобильного приложения на языке программирования Kotlin.
\end{enumerate}

Во <<Введении>>  требуется обосновать актуальность выбранной темы, которая
определяется значимостью ее теоретического и практического решения.
Формулируются цель и задачи практики, определяются объект и предмет
исследования, методология исследования. Целесообразно также охарактеризовать
степень разработанности темы в отечественной и зарубежной литературе,
изложить структуру работы. Этот раздел должен включать в себя краткое
содержание осовных разделов отчета. Введение не должно составлять более 8--10\%
от общего объема отчета (3--5 страниц). Так как <<Введение>> содержит в себе
оприсание постановки задачи и освещает то, как тема разработана ранее другими
авторами, то этот раздел предполагает наличие ссылок на другие источники.
\texttt{Приведем пример того}, как ссылки должны быть
оформлены~\cite{CitekeyArticle, Boreskov2010}.

