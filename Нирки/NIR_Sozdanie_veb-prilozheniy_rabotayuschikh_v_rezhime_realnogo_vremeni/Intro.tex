\intro

На текущий момент, с активным ростом скорости передачи информации внутри информационно-коммуникационных сетей, спрос на обмен информации в режиме реального времени неуклонно растет, информационные системы обмена сообщениями, торговли акциями, онлайн-бронирования, компьютерные игры, требуют возможности не только отправлять данные с клиента, но и получать информацию обо всех изменениях с сервера в режиме реального времени.

Традиционно, все веб-сайты создавались на основе модели запрос/ответ, требуя определенных действий пользователя, для загрузки новой информации, с внедрением технологии AJAX в браузеры, появилась возможность имитировать обновление данных в реальном времени, с помощью цикличных запросов, что позволило создавать сложные веб-приложения с динамичным состоянием \cite{RealtimeWebAppPhp}. На данный момент, в браузерах появились технологии, позволяющие создавать полнодуплексные соединение между клиентом и сервером через одно TCP соединение.

В ходе данного научного исследования будут рассмотрены методы, технические средства и протоколы, позволяющее осуществлять коммуникацию в режиме реального времени, проанализирована их эволюция и перспективы дальнейшего развития.  В качестве практической части, будет спроектировано и реализовано веб-приложение «Мессенджер», позволяющее обмениваться информацией внутри организации в режиме реального времени.

Объектом данного исследования является веб-приложение «Мессенджер», обеспечивающие корпоративную коммуникацию. Предметом исследования являются особенности, методы и средства создания веб-приложений, работающих в режиме реального времени.

Общая цель работы — исследовать этапы, методы и средства создания веб-приложений, работающих в режиме реального времени, спроектировать и осуществить программную реализацию прототипа веб-приложения, обеспечивающие корпоративную коммуникацию, с помощью изученных в результате исследования методов и средств. В ходе достижения общей цели научно-исследовательской работы решаются следующие задачи:

\begin{enumerate} 
  \item Изучение предметной области, а именно концепции приложений, работающих в режиме реального времени, методы и средства разработки приложений, работающих в режиме реального времени.
  
  \item Проектирование веб-приложения «Мессенджер», обеспечивающие корпоративную коммуникацию в режиме реального времени, включающие такие этапы, как сравнение различных средств обеспечения обмена информацией в режиме реального времени, обоснование выбора программных средств, проектирование пользовательского интерфейса.
  
  \item Программная реализация информационной системы «Мессенджер», обеспечивающие корпоративную коммуникацию в режиме реального времени.
\end{enumerate}