\conclusion
В ходе данной учебной практики было рассмотрено что такое коммуникация в режиме реального времени и приложения, работающие в режиме реального времени.  Были рассмотрены основные категории систем реального времени в виде мягких и жестких систем. Были рассмотрены методы, технические средства и протоколы, позволяющее осуществлять коммуникацию в режиме реального времени, в частности, использование цикличных HTTP запросов, WebSocket и WebRTC, проанализирована их эволюция и перспективы дальнейшего развития.

Была спроектирована и реализована в виде прототипа информационная система «Мессенджер», в которой для обмена данными между клиентом и сервером используются рассмотренные технологии коммуникации в режиме реального времени. Была рассмотрена организация процесса разработки информационной системы с использованием методологии DevOps. Подробно описана архитектура серверного приложения с использованием платформы Node.js и метафреймворка Nest.js. Разработана архитектура сетевого взаимодействия между клиентом и сервером на основе протокола WebSocket. Разработана архитектура одностраничного веб-приложения на основе экосистемы Vue.js. Выполнено проектирование пользовательского интерфейса веб-приложения.

В рамках учебной практики, проектно-технологической практики были освоены следующие компетенции:

УК-1. Способен осуществлять поиск, критический анализ и синтез информации, применять системный подход для решения поставленных задач.

Компетенция была сформирована в процессе анализа и поиска современной литературы на английском и русском языках по тематике исследования, с помощью библиографических баз Scopus, WoS, elibrary, ResearchGate, ADS, ЭБС Лань, и др.

УК-4. Способен осуществлять деловую коммуникацию в устной и письменной формах на государственном языке Российской Федерации и иностранном(ых) языке(ах).

Компетенция была сформирована в процессе деловой коммуникации с научным консультантом в письменной и устной формах по вопросам, касающихся учебной практики.

ПК-5. Способен анализировать требования к программному обеспечению
и разрабатывать технические спецификации на программные компоненты.

Компетенция была сформирована в процессе исследования методов, технических средств и протоколов, позволяющее осуществлять коммуникацию в режиме реального времени. Разработке архитектуры сетевого взаимодействия между клиентом и сервером на основе протокола WebSocket.

ПК-6. Способен проводить концептуальное, функциональное и логическое проектирование информационных систем.

Компетенция была сформирована в процессе проектирования архитектуры серверной и клиентской части информационной системы «Мессенджер».

ПК-7. Способен проектировать пользовательские интерфейсы по готовому образцу или концепции интерфейса.

Компетенция была сформирована в процессе проектирования пользовательского интерфейса веб-приложения. Разработки дизайн системы, включающей в себя цветовые схемы и компоненты с различными состояниями.