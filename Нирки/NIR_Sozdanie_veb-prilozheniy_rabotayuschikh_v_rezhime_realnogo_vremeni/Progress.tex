\chapter*{Ход выполнения практики}
\addcontentsline{toc}{chapter}{Ход выполнения практики}

\section*{Задания и отметки о выполнении}

\task{Осуществить поиск информации по предметной области исследования,
основываясь на научной, учебной и учебно-методической литературе, как на русском, так
и на английском языках. Необходимо и спользовать современную литературу на английском
и русском языках по тематике учебной практики, проектно-технологической практики, поиск которой можно осуществлять по библиографическим базам Scopus, WoS, elibrary, ResearchGate, ADS, ЭБС Лань и др.}

\task{Изучить основные концепции веб-приложений, работающих в режиме реального времени.}

\task{Исследовать методы, технические средства и протоколы, позволяющее осуществлять коммуникацию
в режиме реального времени, проанализировать их эволюцию и перспективы дальнейшего развития.}

\task{Выполнить техническое проектирование клиенсткой и серверной части информационной системы  «Мессенджер», использовав технологии коммуникации в режиме реального времени для обмена информацией между клиентом и сервером.}

\task{Выполнить проектирование пользовательского интерфейса разрабатываемого веб-приложения.}

\task{Выполнить программную реализацию прототипа информационной системы «Мессенджер», с возможностью текстового обмена информацией в режиме реального времени.}

\task{Подготовить отчет по практике в соответствии с требованиями нормоконтроля кафедры ИСКМ. Оформить отчет по учебной практики, проектно-технологической практики с использованием набора макрорасширений LATEX.}

\task{Подготовить мультимедийную презентацию и доклад по результатам учебной практики, проектно-технологической практики.}

\clearpage
\section*{Отзыв о работе обучающегося}

В ходе выполнения практики, проектно-технологической практики, Мостовой М. С. изучил концепции разработки веб-приложений, работающих в режиме реального времени. Студент рассмотрел различные технологии, позволяющие осуществлять коммуникацию в режиме реального времени, выделил их достоинства и недостатки, проанализировал их эволюцию и перспективы развития. Спроектировал и реализовал в виде прототипа информационную систему, использующую исследованные технологии коммуникации в режиме реального времени. Выполнил проектирование пользовательского интерфейса веб-приложения. Подготовил отчет в соответствии с требованиями нормоконтроля в издательской системе LATEX. Успешно освоил компетенции, предусмотренные учебным планом. Все задачи, поставленные перед студентом, были выполнены в срок. Считаю, что работа Мостового М. С. заслуживает оценки «отлично».

\progressapprov
