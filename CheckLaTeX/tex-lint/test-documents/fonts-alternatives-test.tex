\documentclass[11pt,a4paper]{article} % Альтернативный допустимый размер шрифта

% ========== ТЕСТ 1: Альтернативный шрифтовой пакет ==========
\usepackage{mathptmx} % Альтернативный пакет, включающий и основной, и математический шрифт

% ========== ТЕСТ 2: Правильные маргины и интервалы ==========
\usepackage[margin=25mm,a4paper]{geometry} % Альтернативные единицы измерения
\usepackage{setspace}
\setstretch{1.5} % Альтернативный способ задания интервала 1.5

% ========== Дополнительные пакеты ==========
\usepackage[T2A]{fontenc}
\usepackage[utf8]{inputenc}
\usepackage[russian]{babel}
\usepackage{amsmath}
\usepackage{graphicx}

\title{Тестовый документ с альтернативными настройками}
\author{Тестовый автор}
\date{\today}

\begin{document}

\maketitle

\section{Введение}

Этот документ демонстрирует альтернативные правильные настройки:
\begin{itemize}
    \item Размер шрифта 11pt (допустимый)
    \item Пакет mathptmx вместо times (включает математический шрифт)
    \item Маргины в миллиметрах
    \item Интервал через setstretch вместо onehalfspacing
\end{itemize}

\subsection{Математические формулы}

Проверим математический шрифт:

$$\sum_{i=1}^{n} x_i = \frac{n(n+1)}{2}$$

$$\lim_{x \to 0} \frac{\sin x}{x} = 1$$

\section{Заключение}

Все настройки в этом документе должны считаться допустимыми, хотя могут выдаваться информационные предупреждения о предпочтительных вариантах.

\end{document} 