\documentclass[10pt]{article} % ОШИБКА: размер шрифта 10pt вместо рекомендуемого 12pt

% ========== ТЕСТ 1: Отсутствие определения шрифта ==========
% ОШИБКА: Не подключен шрифтовой пакет

% ========== ТЕСТ 2: Запрещенный пакет шрифтов ==========
% \usepackage{comic} % ОШИБКА: неподходящий пакет для академических документов

% ========== ТЕСТ 3: Неправильные маргины и интервалы ==========
% \usepackage{a4wide} % ОШИБКА: устаревший пакет
% \usepackage[margin=1cm]{geometry} % ОШИБКА: нестандартный маргин

% ========== ТЕСТ 4: Проблемы с интервалами ==========
\usepackage{setspace}
\singlespacing % ОШИБКА: одинарный интервал вместо 1.5
% \doublespacing % ОШИБКА: двойной интервал вместо 1.5

% ========== ТЕСТ 5: Множественные определения ==========
\linespread{1.3} % ОШИБКА: определение интервала 1
\setstretch{1.4} % ОШИБКА: определение интервала 2 (множественное определение)

% ========== ТЕСТ 6: Прямые команды изменения шрифтов ==========
\renewcommand{\familydefault}{\sfdefault} % ОШИБКА: прямое изменение семейства шрифтов

% ========== Дополнительные пакеты ==========
\usepackage[T2A]{fontenc}
\usepackage[utf8]{inputenc}
\usepackage[russian]{babel}
\usepackage{amsmath}

\title{Тестовый документ с ошибками шрифтов}
\author{Тестовый автор}
\date{\today}

\begin{document}

\maketitle

\section{Основной текст}

Этот документ содержит различные ошибки в настройке шрифтов для тестирования системы проверки.

% ========== ТЕСТ 7: Прямые команды в тексте ==========
{\fontsize{14}{16}\selectfont Этот текст имеет прямое изменение размера шрифта} % ОШИБКА

{\fontfamily{cmr}\selectfont Этот текст использует прямую команду смены семейства шрифтов} % ОШИБКА

\subsection{Математические формулы}

Без согласованного математического шрифта:

$$E = mc^2$$

$$\int_{0}^{\infty} e^{-x} dx = 1$$

\section{Дополнительные изменения интервалов}

\setstretch{2.0} % ОШИБКА: еще одно изменение интервала

Этот текст написан с неправильным интервалом.

\end{document} 