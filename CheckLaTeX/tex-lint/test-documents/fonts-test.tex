\documentclass[12pt,a4paper]{article}

% ========== ТЕСТ 1: Правильное использование шрифтов ==========
% Правильно: подключен рекомендуемый пакет times и размер шрифта 12pt
\usepackage{times}
\usepackage{newtxmath} % согласованный математический шрифт

% ========== ТЕСТ маргинов и интервалов ==========
\usepackage[margin=2.5cm,a4paper]{geometry}
\usepackage{setspace}
\onehalfspacing

% ========== Дополнительные пакеты ==========
\usepackage[T2A]{fontenc}
\usepackage[utf8]{inputenc}
\usepackage[russian]{babel}
\usepackage{amsmath}
\usepackage{graphicx}

\title{Тестовый документ для проверки шрифтов}
\author{Тестовый автор}
\date{\today}

\begin{document}

\maketitle

\section{Основной текст}

Это основной текст документа, написанный с использованием правильно настроенного шрифта Times New Roman размером 12pt и межстрочным интервалом 1.5.

\subsection{Математические формулы}

Проверим согласованность основного и математического шрифтов:

$$E = mc^2$$

$$\int_{0}^{\infty} e^{-x} dx = 1$$

$$\frac{a}{b} = \frac{c}{d}$$

\section{Различные варианты текста}

Обычный текст, \textbf{жирный текст}, \textit{курсивный текст}.

\end{document} 