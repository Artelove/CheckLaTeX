\documentclass{article}
\usepackage[utf8]{inputenc}
\usepackage[russian]{babel}

\begin{document}

\title{Тест проверки окружений со списками}
\maketitle

\section{Правильные примеры}

\subsection{Список после двоеточия (строчные буквы, точка с запятой)}

Рассмотрим основные принципы программирования:
\begin{itemize}
\item простота и ясность кода;
\item модульность архитектуры;
\item повторное использование компонентов.
\end{itemize}

Алгоритм включает следующие этапы:
\begin{enumerate}
\item анализ входных данных;
\item обработка информации;
\item формирование результата.
\end{enumerate}

\subsection{Список после точки (заглавные буквы, точки)}

Были выполнены следующие задачи.
\begin{itemize}
\item Создана база данных.
\item Разработан интерфейс.
\item Проведено тестирование.
\end{itemize}

Получены важные результаты.
\begin{enumerate}
\item Повышена производительность системы.
\item Улучшена стабильность работы.
\item Расширена функциональность.
\end{enumerate}

\section{Примеры с ошибками}

\subsection{Отсутствие пунктуации перед списком}

Основные компоненты системы
\begin{itemize}
\item база данных;
\item веб-интерфейс;
\item API сервис.
\end{itemize}

\subsection{Неправильная пунктуация перед списком}

Необходимо выполнить следующие действия,
\begin{enumerate}
\item установить программное обеспечение;
\item настроить конфигурацию;
\item запустить тестирование.
\end{enumerate}

\subsection{Неправильный регистр после двоеточия}

Основные требования к системе:
\begin{itemize}
\item Высокая производительность;
\item Надежность работы;
\item Простота использования.
\end{itemize}

\subsection{Неправильный регистр после точки}

Были решены следующие проблемы.
\begin{enumerate}
\item исправлены критические ошибки.
\item оптимизирован код.
\item обновлена документация.
\end{enumerate}

\subsection{Неправильные окончания элементов (двоеточие)}

Проект включает следующие модули:
\begin{itemize}
\item модуль аутентификации.
\item модуль обработки данных,
\item модуль отчетности;
\end{itemize}

\subsection{Неправильные окончания элементов (точка)}

Достигнуты следующие цели.
\begin{enumerate}
\item Создан прототип системы;
\item Проведены испытания,
\item Получены положительные результаты;
\end{enumerate}

\subsection{Пустые элементы списка}

Структура проекта:
\begin{itemize}
\item 
\item модуль данных;
\item модуль интерфейса.
\end{itemize}

\subsection{Отсутствие текста перед списком}

\begin{enumerate}
\item первый элемент;
\item второй элемент;
\item третий элемент.
\end{enumerate}

\subsection{Только пробелы перед списком}

   
\begin{itemize}
\item элемент один;
\item элемент два;
\item элемент три.
\end{itemize}

\subsection{Неалфавитные символы в начале элементов}

Основные компоненты:
\begin{itemize}
\item 1-й компонент;
\item 2-й компонент;
\item 3-й компонент.
\end{itemize}

\subsection{Тест различных типов списков}

\subsubsection{Description список}

Основные термины:
\begin{description}
\item[API] интерфейс программирования приложений;
\item[SDK] набор средств разработки;
\item[IDE] интегрированная среда разработки.
\end{description}

\subsubsection{List окружение}

Список задач:
\begin{list}{•}{\leftmargin=1cm}
\item задача номер один;
\item задача номер два;
\item задача номер три.
\end{list}

\section{Смешанные случаи}

\subsection{Вложенные списки}

Архитектура системы:
\begin{enumerate}
\item фронтенд:
    \begin{itemize}
    \item пользовательский интерфейс;
    \item система навигации;
    \end{itemize}
\item бэкенд:
    \begin{itemize}
    \item API сервер;
    \item база данных;
    \end{itemize}
\item инфраструктура.
\end{enumerate}

\end{document} 