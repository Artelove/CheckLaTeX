\documentclass[12pt,a4paper]{article}

% Пакеты для работы с русским языком
\usepackage[utf8]{inputenc}
\usepackage[russian]{babel}
\usepackage[T2A]{fontenc}

% Основные пакеты
\usepackage{amsmath,amssymb,amsthm}
\usepackage{geometry}
\usepackage{graphicx}
\usepackage{hyperref}

% Настройка геометрии страницы
\geometry{
    left=2cm,
    right=2cm,
    top=2cm,
    bottom=2cm
}

\title{Тестовый документ для CheckLaTeX}
\author{Автор документа}
\date{\today}

\begin{document}

\maketitle

\section{Введение}

Это тестовый LaTeX документ для демонстрации работы расширения CheckLaTeX. 
В этом документе специально оставлены некоторые ошибки для тестирования анализатора.

\section{Проблемы с кавычками}

Здесь используются "неправильные" кавычки вместо правильных «кавычек».
Также встречаются 'одинарные кавычки' вместо „правильных".

\section{Проблемы с тире}

В этом тексте используется дефис - вместо тире.
Правильно должно быть так — с использованием тире.

\section{Математические формулы}

Пример формулы:
\begin{equation}
    E = mc^2
\end{equation}

Inline формула: $a^2 + b^2 = c^2$.

\section{Списки}

\begin{itemize}
    \item Первый элемент списка
    \item Второй элемент списка
    \item Третий элемент списка
\end{itemize}

\section{Таблица}

\begin{table}[h]
\centering
\begin{tabular}{|c|c|c|}
\hline
Заголовок 1 & Заголовок 2 & Заголовок 3 \\
\hline
Данные 1 & Данные 2 & Данные 3 \\
\hline
Данные 4 & Данные 5 & Данные 6 \\
\hline
\end{tabular}
\caption{Пример таблицы}
\label{tab:example}
\end{table}

\section{Заключение}

Этот документ содержит различные элементы LaTeX для тестирования возможностей 
анализатора CheckLaTeX. Расширение должно обнаружить проблемы с кавычками и тире.

\end{document} 